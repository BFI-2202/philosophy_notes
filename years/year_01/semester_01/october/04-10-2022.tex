\documentclass{article}
\usepackage[utf8]{inputenc}

\usepackage[T2A]{fontenc}
\usepackage[utf8]{inputenc}
\usepackage[russian]{babel}

\title{Философия}
\author{Лисид Лаконский}
\date{October 2022}

\begin{document}

\maketitle
\tableofcontents
\pagebreak

\section{Философия - 04.10.2022}

\subsection{Философские категории}

\subsubsection{Причина и следствие}

Причина и следствие – философские категории, выражающие одну из форм всеобщей связи явлений.

Причина (лат. causa) обычно мыслится как явление, действие которого производит, определяет или вызывает другое явление; последнее называют следствием. 

\subsubsection{Сущность и явление}

Сущность и явление — философские категории, отражающие всеобщие формы предметного мира и его познание человеком.

Сущность — это внутреннее содержание предмета, выражающееся в единстве всех многообразных и противоречивых форм его бытия

Явление — то или иное обнаружение (выражение) предмета, внешние формы его существования.

В мышлении категории сущности и явления выражают переход от многообразия наличных форм предмета к его внутреннему содержанию и единству. Постижение сущности предмета составляет задачу науки.

\subsubsection{Содержание и форма}

Содержание понятия — это совокупность существенных и отличительных признаков предмета, качества или множества однородных предметов, отражённых в этом понятии.

Например, содержанием понятия «коррупция» является совокупность двух существенных признаков: «сращивание государственных структур со структурой преступного мира» и «подкуп и продажность общественных и политических деятелей, государственных чиновников и должностных лиц».

Форма - понятие философии, определяемое соотносительно к понятиям содержания и материи. В соотношении с содержанием, форма понимается как упорядоченность содержания — его внутренняя связь и порядок.

\subsubsection{Необходимость и случайность}

Необходимость и случайность – соотносительные философские понятия;

Необходимым называют явление, однозначно детерминированное определенной областью действительности, предсказуемое на основе знания о ней и неустранимое в ее границах;

Случайным называют явление, привнесенное в эту область извне, не детерминированное ею и, следовательно, не предсказуемое на основе знания о ней.

\subsubsection{Возможность и действительность}

Возможность и действительность - философские категории, логически описывающие движение, способ существования материи во времени.

Действительность — это то, что уже возникло, существует.

Возможность — это то, что может возникнуть и существовать при определённых условиях, стать действительностью.

\end{document}
