\documentclass{article}
\usepackage[utf8]{inputenc}

\usepackage[T2A]{fontenc}
\usepackage[utf8]{inputenc}
\usepackage[russian]{babel}

\title{Философия}
\author{Лисид Лаконский}
\date{October 2022}

\begin{document}

\maketitle
\tableofcontents
\pagebreak

\section{Философия — 27.10.2022}

\subsection{Философия эпохи Возрождения}

\begin{flushleft}

\subsubsection{Предпосылки}

\begin{enumerate}
    \item \textbf{Смена экономического устройства общества} - капитализм на смену феодализма (корпоративное общество) - изменение представлений о мире и о месте человека в мире - индивидуализм
    \item \textbf{Религиозная реформация} - секуляризация общества - потеря влияния католической церкви - рождение протестантизма (индивидуализм) - Мартин Лютер: sola scriptura, непознаваемость бога - познание природы
    \item \textbf{Зарождение экспериментального естествознания}
\end{enumerate}

\subsubsection{Общая характеристика}

Человек мыслит себя богом и воспринимает себя \textbf{творцом природы}.

Появление ордена франкмасонов (свободных каменщиков): природа - храм, а мы в нем строители.

Культ человека-творца. Отражение в архитектуре: многочисленные дворцы, виллы, резиденции.

Преодоление заблуждений античности.

\hfill

\textbf{Николай Кузанский} - появление идеи континуума, \textbf{Джордано Бруно} - пантеистическая концепция - появление их множества, \textbf{Джовани Боккаччо} - ,Декамерон', \textbf{Данте Алигьери} - ,Божественная комедия'

Развитие наук - \textbf{Галилео Галилей} - основатель экспериментальной физики

\textbf{Николай Коперник} - гелиоцентрическая система - не новая идея, исток - античность. Орбиты - окружность, отсутствие какой-либо аргументации - считалось само собой разумеющимся - из представлений античности.

Астрономическая теория эпициклов - дополнение теории Коперника, призванная устранить ее недостатки - расчеты не сходились с наблюдениями.

\end{flushleft}

\end{document}
