\documentclass{article}
\usepackage[utf8]{inputenc}

\usepackage[T2A]{fontenc}
\usepackage[utf8]{inputenc}
\usepackage[russian]{babel}

\title{Философия}
\author{Лисид Лаконский}
\date{October 2022}

\begin{document}

\maketitle
\tableofcontents
\pagebreak

\section{Философия — 27.10.2022}

\begin{flushleft}

Средние века называются средними, потому что находятся между античностью и ренессансом.

\hfill

15-й век - осознание ухода греко-римской культуры, ее возрождение.

16-й век - придумали разделение на средневековье и новое время.

\hfill

Текущее время - постмодерн (забыть ,новейшее время').

\subsection{Доклады}

\subsubsection{Основные характеристики мировоззрения Средневековья}

Господство религии, богоцентрического мировоззрения.

Суть религиозного мировоззрения состояла в противопостоянии жизни мирской и духовной.

Глубочайшее влияние религии на науку. Подчинение философии религия.

Корпоративность общества. Ограниченность корпорацией. Разрыв между элитой и простым народом.

Отказ от античной культурной традиции - но начали формироваться светские университеты - повышение интереса к античности.

\hfill

Нельзя сказать, что характерен упадок - это важный период для становления нового времени.

Резюмируя, характеристики мировоззрения:

\begin{enumerate}
    \item Господство религии
    \item Корпоративность общества
    \item Отказ от античной культурной традиции
\end{enumerate}

\textbf{Комментарии.}

\hfill

Инквизиция начала жечь людей только тогда, когда церковь стала терять свою власть.

Корпоративность - невозможна жизнь вне корпорации. Сейчас - индивидуализм.

\subsubsection{Патристика}

Патристика - один из нескольких этапов средневековой философии. Относят к периоду со второго по восьмой век.

\textbf{Патристика} - философское и религиозное учение, основа - тексты, написанные философами, священнослужителями - отцами церкви.

Например, \textbf{Афанасий Великий}, \textbf{Григорий Богослов}, \textbf{Иоанн Златоуст}.

Основа христианского богословия. Упорядочивание знаний о христианстве.

Задача философов - \textbf{анализ священных текстов}, рассуждения на их основе. Разъяснение значения различных религиозных догматов.

Неоспоримый источник знания - Библия.

Три исторических этапа:

\begin{enumerate}
    \item Ранний (2-й - 3-й век) - сильно религиозные тексты, две категории философов: апостольские отцы, защита христианства (апологеты); другая группа - человек и его место в мире
    \item Второй этап (3-й - 4-й век) - появление первых догм, расширение патристики, расширение влияния.
    \item Третий этап - закрепление патристики как части христианской истории, сочинения становятся канонами.
\end{enumerate}
\textbf{ранняя} (2-й - 3-й век), характерные черты - сильно религиозные тексты, две категории философов: апостольские отцы, защита христианства (апологеты); другая группа - человек и его место в мире

\hfill

\textbf{Резюме.} Патристика заложила основы христианства, повлияла на его дальнейшее развитие. Основа для схоластики.

\end{flushleft}

\end{document}