\documentclass{article}
\usepackage[utf8]{inputenc}

\usepackage[T2A]{fontenc}
\usepackage[utf8]{inputenc}
\usepackage[russian]{babel}

\title{Философия}
\author{Лисид Лаконский}
\date{October 2022}

\begin{document}

\maketitle
\tableofcontents
\pagebreak

\section{Философия — 13.10.2022}

\subsection{Классическая эллинская философия}

\begin{flushleft}

\textbf{Классическая эллинская философия} - пятый-четвертый век до нашей эры. 

\subsubsection{Предпосылки формирования}

Шестой век до нашей эры - \textbf{судебная реформа Солона}.

До судебной реформы - \textbf{суд архонтов}, градоправителей - единоличное решение.

Судебная реформа Солона - \textbf{суд гелиадов (гелиэя)} - выборные судьи. Обвинитель и обвиняемый выступали, доказывали свою правоту.

\hfill

Правом голоса в Древней Греции обладали лишь жители полисов мужского пола. Больше половины жителей, условно, Афин, не имели права голоса.

\hfill

\textbf{Особенности афинской демократии}: неравновесные голоса у граждан (зависит от количества земли в собственности).

\hfill

\textbf{Нормы публичного обвинения} - народ Афин в качестве обвинителя, не конретный человек.

\hfill

Судебная реформа стимулировала развитие риторики и навыков аргументации. После реформ в Афинах открылись школы риторики - преобразовались в философские школы. Большое количество. Представители этих школ - в будущем \textbf{софисты}. Брали очень большие деньги за обучение.

\hfill

\textbf{На момент пятого века:} суд гелиадов, множество школ риторики, диверсификация образования.

\subsubsection{Философия Сократа}

\textbf{Сократ} не оставил после себя ни одного письменного сочинения - не писал принципиально - его философию знаем через его учеников. Полулегендарная личность.

\hfill

\textbf{Общая характеристика сократической философии.} Понятийная философия - не созерцательная. Поворот от философии созерцательной к философии понятийной. Изменился предмет интереса философии - \textbf{изобретение западной философской мысли}. Абстрактные предметы: добродетель, отвага, дружба.

\hfill

Главной темой философии становится человек. Но главное - методологической поворот, изменение греческого мышления. Становится \textbf{строго абстрактным и понятийным}.

\hfill

\textbf{Философский метод Сократа}. Не оставил ни одного письменного сочинения - был уверен, что философия возможна только как беседа одной души с другой (или с самой собой). Философия - \textbf{живой разговор}.

Метод его бесед - \textbf{майевтический (маевтика)} - буквально "повивальное искусство".

Раскрытии истины путём последовательных вопросов, через "испытание". \textbf{Искусство наводящих вопросов}.

\hfill

Все известно через учеников: \textbf{Платона}, \textbf{Ксенофонта} и так далее.

\hfill

Сократ \textbf{был одним из учителей риторики}. Но не брал денег за обучение. Был женат, были дети, воевал. Возможно, имел психическое расстройство.

\hfill

Сократ \textbf{был казнен, приговорен к самоубийству}. Народ Афин обвинил его в том, что он совращает молодежь, оскорбляет богов - в итоге прокатило.

Произведения: "Апология Сократа", "Критон".

Две версии, почему он принял смерть: одна изложена в диалоге "Критон" - \textbf{возжелал узнать, что будет после смерти}. Вторая версия - отказался уезжать из полиса, потому что это \textbf{стало бы изгнанием}.

\hfill

Легендарные последние слова: попросил принести дар в жертву Аполлону, какое-то очень малое подношение.

\hfill

\textbf{Как попал под обвинение?} Учение за бесплатно не понравилось софистам, бравшим очень большие деньги. У него была очень мощная рекламная кампания - дельфийский оракул признал его \textbf{мудрейшим из людей}. Казнь Сократа - возможно, \textbf{проявление неконкуретной борьбы}. Софисты избавились от конкурента.

\subsubsection{Философия Платона}

\textbf{Платон} - самый известный ученик \textbf{Сократа}. Во времена Древней Греции был овеян множеством мифов.

\hfill

Платон - \textbf{прозвище}, буквально широкий, не настоящее имя. Настоящее имя - \textbf{Аристокл}. Гражданин Афин, родом из очень богатой семьи, было очень много свободного времени.

\hfill

\textbf{Основатель Платоновской Академии} - слово "академия" пошло от него. "Академия" было именем собственным здания. Пошло от имени афинского героя Академа.

Являлся своего рода \textbf{дискуссионным клубом} - поощрались споры, обсуждения - место для свободного спора.

Помимо философских дискуссий занимались \textbf{развитием наук}: всего подряд, чем занимались греки.

\hfill

Платон \textbf{оставил большой корус сочинений}. Его произведения - \textbf{диалоги}. Главный лирический герой всех их - Сократ. Майевтический метод.

\hfill

Платон \textbf{не оставил цельной философской системы}. Его диалоги \textbf{часто противоречат друг другу}. Менялись взгляды, убеждения - философия трансформировалась. Сложно говорить об общей философской системы.

Каждый диалог - \textbf{отдельная философская система}.

\hfill

\textbf{Реконструкция философской системы}. Платон - представитель \textbf{объективного идеализма} - мир существует независимо от нас - у него есть абстрактные основания. Есть элементы \textbf{субъективного идеализма}.

\hfill

Лирическое отступление - \textbf{основной вопрос философии}:

\begin{enumerate}
    \item Материализм
    \item Идеализм - у мира есть идеальные основания
    \begin{enumerate}
        \item Объективный - идеальное основание существует отдельно и независимо от нас
        \item Субъективный - у мира идеальные основания, но эти основания мы обнаруживаем в себе - мы сами творим мир.
    \end{enumerate}
\end{enumerate}

\hfill

\textbf{Онтология - объективный идеализм.} Платон полагал, что мир - не подлинное бытие, \textbf{иллюзорное бытие}, так как предметов когда-то не было и когда-нибудь их не станет. Уровень бытийности предметов очень низкий - существуют короткий промежуток времени. Времени несуществования гораздо больше.

Вещи находятся в становлении, наш мир - \textbf{мир становления}. Вещи появляются и уничтожаются.

Платон полагал, что подлинное бытие неучтожимо, вечно. Подлинное бытие - \textbf{царство идей} - реальный, настоящий мир.

\hfill

\textbf{Идея} - это вещь, взятая в ее совершенстве. Образец для всех вещей в мире. Шаблон. Общее между вещами.

Общее между столами - идея стола. Образец для всех столов в мире.

Идеи существуют для всего в нашем мире.

\hfill

Мощнейшее влияние на развитие западной философской мысли. Природа теоретического знания.

\hfill

Очень большое влияние на математику. \textbf{Числа - тоже, конечно, идеи}. Числа обладают самостоятельным существованием в царстве идей.

\hfill

\textbf{Иерархия идей}. Нижний уровень - \textbf{идеи вещей}, выше уровень - \textbf{идеи чисел}, дальше - \textbf{идеи отношений между вещами}, далее - \textbf{идеи добродетелей} (мудрость, отвага, справедливость).

\textbf{Идея блага - высшая идея}.

\subsubsection{Философия Аристотеля}

\textbf{Аристотель} - ученик Платона. Аристотель был \textbf{не согласен с концепцией идей}:

\begin{enumerate}
    \item Идеи никак не помогают в познании мира, \textbf{идеи - лишняя сущность}
    \item Почему не существует идей у идей?
\end{enumerate}

Принято считать \textbf{материалистом}. Но в современном смысле слова \textbf{материалистом не был}.

\hfill

\textbf{Метафизика. Учение о причинах.} Аристотель полагал, что у всего в мире существуют свои причины. Считал, что теоретическое знание о вещах необходимо применять к самим вещам.

\textbf{Мир постоянно меняется} - становление. У каждого изменения должна быть причина. Все в мире находится в движении, и у каждого движения должна быть причина.

\hfill

\textbf{Производящая причина} - происхождение вещи. Например, производящая причина стола - мебельная фабрика.

\textbf{Материальная причина} - то, из чего сделана вещь. Например, материальная причина стола - дерево.

\textbf{Целевая причина} - для чего вещь нужна. Например, стол нужен для того, чтобы на него что-то ставить.

\textbf{Формальная причина} - что это за вещь. Например, стол - это штука из ножек, столешницы и так далее.

\hfill

Аристотель считал, что все создано из \textbf{материи}, но в чистом виде ее нет. Материя без формы - небытие.

Форма - и есть \textbf{формальная причина}.

\hfill

Совокупность четырех причин - \textbf{эйдос} в философии Аристотеля - \textbf{причинно-целевая конструкция вещи}. Не имеет смысла в отрыве от вещи, \textbf{строго привязан к вещи}. Вещь без эйдоса - небытие. Эйдос без вещи не имеет смысла.

\hfill

Эйдос всех эйдосов - \textbf{ум-перводвигатель}. Причина всех причин, самая первая причина. То, что привело самым первым все в движение.

\hfill

\textbf{Характеристики перводвигателя}. Сам перводвигатель \textbf{неподвижен}, так как если бы он был в движении - он был бы уже не перводвигателем. Перводвигатель \textbf{вечен}, иначе бы у его появления была причина. Перводвигатель \textbf{разумен}, ум-перводвигатель, только разум может стать причиной другого разума. 

Перводвигатель - \textbf{эйдос мира}. Причина всех других эйдосов. Высший эйдос.

\hfill

\textbf{Теория не имеет смысла в отрыве от предмета}. Попытка описать мир в его становлении, от чего отказался Платон. Считал, что причина должна описывать мир.

\end{flushleft}

\end{document}