\documentclass{article}
\usepackage[utf8]{inputenc}

\usepackage[T2A]{fontenc}
\usepackage[utf8]{inputenc}
\usepackage[polutonikogreek, english,russian]{babel}
\usepackage{csquotes}

\title{Философия}
\author{Лисид Лаконский}
\date{November 2022}

\begin{document}

\maketitle
\tableofcontents
\pagebreak

\section{Философия - 10.11.2022}

\subsection{Эмпиризм в философии нового времени}

\subsubsection{Эмпиризм в философии}

\begin{flushleft}

Термин «эмпиризм» происходит от греческого слова \begin{greektext}ἐμπειρία\end{greektext}, что переводится как «опыт». Эмпиризм объясняет окружающий мир и его явления с позиций опыта, полагая, что знание можно добыть исключительно опытным путем.

Представители эмпиризма в философии полагают, что, во-первых, знание можно добыть исключительно из опыта, во-вторых, знание, теория, догадка или предположение могут считаться верными, лишь когда они подтверждены практическим опытом.

\subsubsection{История}

Протагор (485-410): «Как мы чувствуем что-либо, так это и есть на самом деле».

Но основоположником эпиризма в философии как полноценного философского течения стал Фрэнсис Бэкон (1561-1626)

\subsubsection{Эмпиризм Ф. Бэкона}

В основе научного знания, по Бэкону, лежат эксперимент и индукция как метод обработки полученных данных. Бэкон предлагал двигаться от частных, полученных в ходе эксперимента наблюдений к более общим выводам.

\subsubsection{Формы эмпиризма}

Различают две формы эмпиризма:

\begin{enumerate}
    \item Имманентный эмпиризм — попытка объяснить суть знания как совокупность представлений и ощущений, полученных опытным путем
    \item Трансцендентный эмпиризм — попытка объяснить происхождение знания как продукта взаимодействия сознания с окружающим миром
\end{enumerate}

Другими словами, если \textbf{имманентный эмпиризм} конечным продуктом считает совокупность представлений и ощущений, \textbf{трансцендентный эмпиризм} конечным продуктом считает то, что получилось в результате осознания полученного опыта.

\subsection{Т. Гоббс: механицизм, теория общественного договора}

\subsubsection{Механицизм}

Философия — наука о телах; геометрия Эвклида, физика Галилея — образцы этой науки

Два элемента, которыми объясняется действительность — основа механистического материализма Гоббса:

\begin{enumerate}
    \item Тело
    \item Движение тела в пространстве
\end{enumerate}

Основа материальных тел — Бог.

Процесс познания: объект вызывает у субъекта ощущение, которое представляет собой некое движение; движение же при взаимодействии с чувственной природой субъекта порождает реакцию на движение, то есть новое (ответное) движение.

\subsubsection{Теория общественного договора}

Теория общественного договора — \textbf{концепция происхождения государства}.

Суть — люди договорились принять соглашение о превращении своего \textbf{«естественного состояния»} в \textbf{«состояние гражданское»}. 

\hfill

Т. Гоббс  в своем труде «Левиафан» указывал на то, что люди, находясь в своем «естественном состоянии», прибывали в положении «войны всех против всех».

Естественное состояние общества Гоббс определил как \textbf{исключительно индивидуалистическое}, поскольку в  таком  состоянии люди  равны  между  собой,  свободны  и  независимы  друг от  друга, а  по  своей естественной природе  к  тому  же являются  существами  эгоистичными,  всегда  действующими  из соображений собственной выгоды и безопасности.

И для того, чтобы обезопасить себя, они решили заключить договор, по которому \textbf{отказываются от полной индивидуальной свободы в пользу государства}, обеспечивающего социальный порядок.

Задачами такого государства должны быть \textbf{безопасность, стабильность и процветание} его народа.

\subsection{Ф. Бэкон: метод индукции, учение об идолах}

\subsubsection{Метод индукции}

\paragraph{Введение} Индуктивный метод познания у Бэкона значительно отличался от того, коим пользовались ранее. В целом, суть была одна — переход от частного к общему, но важно было само понимание этих двух, путь от одного к другому.

Индукция должна была разделять природу посредством \textbf{разграничений и исключений}. И, после достаточного количества отрицательных суждений, заключать о положительном. Огромное значение уделялось при этом \textbf{аксиомам — законам, формам отличия вещей}. Бэкон утверждал, что следует переходить от частностей к меньшим аксиомам, а затем к средним и, наконец, к общим.

Основное отличие такой индукции от предшествующей — \textbf{сосредоточенность на опыте и частных случаях, выведение аксиом}, строго исходя только из них, тогда как предыдущая индукция лишь бегло их касалась, строя гипотезы практически из ничего.

\paragraph{Аксиомы}

Бэкон считал, что правильно открытые и установленные \textbf{аксиомы} влекут за собой многочисленные ряды практических приложений. Эти аксиомы открывают исследователю \textbf{формы или истинные отличия вещей}.

Для наук следует ожидать пользы тогда, когда исследование восходит по непрерывным ступеням — от частностей к меньшим аксиомам и затем к средним, одна выше другой, и наконец к самым общим.

\paragraph{Формы}

По Бэкону, в природе не существует ничего действительно, помимо \textbf{единичных тел, осуществляющих сообразно с законом отдельные чистые действия}.

Этот закон и его подразделения Бэкон называет \textbf{формами}.

\paragraph{Исследование форм}

Бэкон неоднократно предупреждает, что никто успешно не отыщет природу вещи в самой вещи — исследование \textbf{должно быть расширено до более общего}, поскольку то, что в одних вещах считается скрытым, в других имеет явную и обычную природу.

Исследование форм происходит следующим образом:

\begin{enumerate}
    \item Сначала нужно для каждой данной природы представить в таблице все известные примеры, сходящиеся в этой природе, хотя бы и посредством самых различных материй. Эта таблица называется таблицей присутствия
    \item Во-вторых, должно представить разуму примеры, которые лишены данной природы (в предметах наиболее родственных тем, в которых данная природа присутствует), так как форма так же должна отсутствовать там, где отсутствует природа, как и присутствовать там, где она присутствует. Это таблица отсутствия в ближайшем
    \item В-третьих, должно представить разуму примеры, в которых исследуемая природа присутствует в большей и в меньшей степени. Это возможно или посредством сопоставления роста и уменьшения этого свойства в одном и том же предмете, или посредством сравнения его в различных предметах. Это таблица степеней
\end{enumerate}

Эти три таблицы представляют исследователю примеры. За этим следует \textbf{сама индукция}:

\begin{displayquote}
Первое дело истинной индукции есть отбрасывание отдельных природ, которые не встречаются в каком-либо примере, где присутствует данная природа, или встречаются в каком-либо примере, где отсутствует данная природа, или встречаются растущими в каком-либо примере, где данная природа убывает, или убывают, когда данная природа растёт. После сделанного должным образом исключения останется положительная и хорошо определённая форма.
\end{displayquote}

\subsubsection{Учение об идолах}

\textbf{Идолы}, согласно Фрэнсису Бэкону — причины, которые \textbf{препятствуют человеку и человечеству получить истинное знание}:

\begin{enumerate}
    \item Врожденные заблуждения:
    \begin{enumerate}
        \item Идолы рода — преломление познания через культуру человека (рода в целом) — человек осуществляет познания, находясь в рамках общечеловеческой культуры, и это снижает истинность знания
        \item Идолы пещеры — влияние личности конкретного человека на процесс познания — личность человека (его предрассудки, заблуждения — «пещера») отражается в конечном результате познания
    \end{enumerate}
    \item Приобретенные заблуждения:
    \begin{enumerate}
        \item Идолы рынка — неправильное, неточное употребление понятийного аппарата: слов, дефиниций
        \item Идолы театра — влияние существующей философии на процесс познания — старая философия мешает проявлять новаторский подход, направляет познание не всегда в нужное русло
    \end{enumerate}
\end{enumerate}

\subsection{Д. Локк: «tabula rasa», естественные права человека}

\subsubsection{tabula rasa}

В философии Локка tabula rasa была теорией, согласно которой \textbf{при рождении (человеческий) разум представляет собой "чистый лист"} без правил обработки данных, и что данные добавляются, а правила обработки формируются исключительно \textbf{на основе чувственного опыта}.

\hfill

В понимании Локка tabula rasa означала, что \textbf{ум человека рождается пустым}, а также подчеркивает свободу людей создавать свою собственную душу. Индивиды \textbf{свободны определять содержание своего характера}, но \textbf{основная идентичность как члена человеческого вида не может быть изменена}.

\subsubsection{Естественные права человека}

Локк полагал, что еще до того, как появилось государство, \textbf{человеку природой были даны естественные права: на жизнь, свободу и собственность} — часть человека. Идея состояла в том, что люди не рождаются королями или рабами, — они \textbf{рождаются свободными, с правом распоряжаться самостоятельно своей жизнью и собственностью}.

\hfill

Человек \textbf{не может бросить эти права или передать другому}, также как и \textbf{не может их у кого-то отобрать}. Тому есть две причины: во-первых, по мнению Локка, \textbf{человек находится в собственности у Бога}, который его создал; во-вторых, свобода не дает человеку право делать все, что угодно.

\hfill

Государство было создано для того, чтобы \textbf{урегулировать права людей}, данные природой — \textbf{человек не может не быть частью социума} и на подсознательном уровне \textbf{дает свое внутреннее согласие} на такие условия.

\subsection{Рационализм в философии нового времени: общая характеристика, основные черты}

Эмпиризм в какой-то мере стал стимулом к развитию такого направления философии как рационализм. В отличие от эмпиризма, рационализм объявлял \textbf{источником знания разум}, а \textbf{основным методом познания – дедукцию}.

Основатель рационализма — Рене Декарт (1596-1650) — заявлял, что восприятие может обмануть, а полученные в ходе эксперимента данные не всегда прямо указывают на причину изучаемого явления.

Поэтому \textbf{любое наблюдение, предположение, знание должны пройти проверку разумом}, в ходе которой следует \textbf{отбросить все авторитеты и подвергнуть сомнению все}, даже кажущиеся незыблемыми истины.

\hfill

\textbf{Но} эмпиризм и рационализм в философии нельзя считать строго противоположными течениями. Любой эксперимент подлежит осмыслению, методы индукции и дедукции абсолютно равноправны и могут применяться в рамках даже одного исследования для проверки выводов и поиска неточностей.

\subsection{Р. Декарт: учение о методе, «cogito ergo sum»}

\subsubsection{Учение о методе}

Рационализм Декарта основывается на идеи \textbf{всеобщей математизации научного познания}

\paragraph{Основные положения}

Суть метода Декарта сводится к \textbf{двум положениям}:

\begin{enumerate}
    \item В познании следует отталкиваться от \textbf{интеллектуальной интуиции}, рождающейся в здравом уме, знания настолько простого и отчетливого, что оно \textbf{не вызывает никакого сомнения}
    \item Разум должен из этих интуитивных воззрений \textbf{на основе дедукции} вывести все необходимые следствия
\end{enumerate}

\paragraph{Дедукция}

Декарт сформулировал следующие \textbf{3 правила} дедуктивного метода:

\begin{enumerate}
    \item Во всяком вопросе \textbf{должно содержаться неизвестное}
    \item Это неизвестное \textbf{должно иметь какие-то характерные особенности}, чтобы исследование было направлено на постижение именно этого неизвестного
    \item В вопросе также \textbf{должно содержаться нечто известное}
\end{enumerate}

\paragraph{Правила познания истины}

\begin{enumerate}
    \item Истинно то, что очевидно и отчетливо. Постичь истину до конца можно только интеллектуальной интуицией
    \item Сложные идеи нужно раскладывать на простые
    \item Идти от простых истин к сложным
    \item Расчлененные простые истины связывать с помощью интуиции
\end{enumerate}

\paragraph{Классы идей}

\begin{enumerate}
    \item Потусторонние — появляющиеся извне
    \item Фантазии — образуемые человеком
    \item От Бога — врожденные идеи
\end{enumerate}

\paragraph{Бытие}

Бытие образует две субстанции:

\begin{enumerate}
    \item Материальную, ее атрибут — протяженность
    \item Духовную, ее атрибут — мышление
\end{enumerate}

Интеллектуальная интуиция у Декарта \textbf{начинается с сомнения} — помогает \textbf{избавиться от предрассудков} (идолов по Бэкону).

\hfill

Поставив под сомнение достоверность всех наших представлений о мире, мы можем допустить, писал Декарт, что положение «Я мыслю, следовательно, я существую» и есть представление о том, что \textbf{мышление, независимо от его содержания и объектов, демонстрирует реальность мыслящего субъекта} и \textbf{является той первичной исходной интеллектуальной интуицией}, из которой выводятся все знания о мире

\hfill

Принцип ясности и отчетливости знания гарантирован у Декарта существованием \textbf{Бога — совершенного и всемогущего, вложившего в человека свет разума}, врожденные идеи — к ним Декарт относил идею Бога как существа всесовершенного, идеи чисел и фигур.

\hfill

Согласно Декарту, \textbf{источником заблуждений не может быть разум} сам по себе. Заблуждения есть продукт \textbf{злоупотребления человеком присущей ему свободой воли}. Заблуждения возникают тогда, когда бесконечно свободная воля переступает границы конечного человеческого разума.

Но Декарт верит в \textbf{неограниченные возможности человеческого разума в деле познания} всей окружающей его действительности.

\subsubsection{Cogito ergo sum}

Cogito ergo sum — «я мыслю, следовательно, я существую» — философское утверждение Рене Декарта, \textbf{фундаментальный элемент западного рационализма} Нового времени.

\hfill

Это утверждение Декарт выдвинул как \textbf{первичную достоверность, истину, в которой невозможно усомниться} — и с которой, следовательно, можно \textbf{начинать отстраивать здание достоверного знания}.

\hfill

Аргумент \textbf{не следует понимать как умозаключение}; напротив, его \textbf{суть — в очевидности}, самодостоверности моего существования как мыслящего субъекта: всякий акт мышления (и шире — всякое представление, переживание сознания, ибо cogito не ограничивается мышлением) обнаруживает — при рефлексивном взгляде на него — меня, мыслящего, осуществляющего этот акт. \textbf{Аргумент указывает на самообнаружение субъекта в акте мышления} (сознания): я мыслю — и, созерцая своё мышление, обнаруживаю себя, мыслящего, стоящего за его актами и содержаниями.

\end{flushleft}

\end{document}