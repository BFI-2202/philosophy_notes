\documentclass{article}
\usepackage[utf8]{inputenc}

\usepackage[T2A]{fontenc}
\usepackage[utf8]{inputenc}
\usepackage[polutonikogreek, english,russian]{babel}
\usepackage{csquotes}

\title{Философия}
\author{Лисид Лаконский}
\date{November 2022}

\begin{document}

\maketitle
\tableofcontents
\pagebreak

\section{Философия - 10.11.2022}

\subsection{Эмпиризм в философии нового времени}

\subsubsection{Эмпиризм в философии}

\begin{flushleft}

Термин «эмпиризм» происходит от греческого слова \begin{greektext}ἐμπειρία\end{greektext}, что переводится как «опыт». Эмпиризм объясняет окружающий мир и его явления с позиций опыта, полагая, что знание можно добыть исключительно опытным путем.

Представители эмпиризма в философии полагают, что, во-первых, знание можно добыть исключительно из опыта, во-вторых, знание, теория, догадка или предположение могут считаться верными, лишь когда они подтверждены практическим опытом.

\subsubsection{История}

Протагор (485-410): «Как мы чувствуем что-либо, так это и есть на самом деле».

Но основоположником эпиризма в философии как полноценного философского течения стал Фрэнсис Бэкон (1561-1626)

\subsubsection{Эмпиризм Ф. Бэкона}

В основе научного знания, по Бэкону, лежат эксперимент и индукция как метод обработки полученных данных. Бэкон предлагал двигаться от частных, полученных в ходе эксперимента наблюдений к более общим выводам.

\subsubsection{Формы эмпиризма}

Различают две формы эмпиризма:

\begin{enumerate}
    \item Имманентный эмпиризм — попытка объяснить суть знания как совокупность представлений и ощущений, полученных опытным путем
    \item Трансцендентный эмпиризм — попытка объяснить происхождение знания как продукта взаимодействия сознания с окружающим миром
\end{enumerate}

Другими словами, если \textbf{имманентный эмпиризм} конечным продуктом считает совокупность представлений и ощущений, \textbf{трансцендентный эмпиризм} конечным продуктом считает то, что получилось в результате осознания полученного опыта.

\subsection{Т. Гоббс: механицизм, теория общественного договора}

\subsubsection{Механицизм}

Философия — наука о телах; геометрия Эвклида, физика Галилея — образцы этой науки

Два элемента, которыми объясняется действительность — основа механистического материализма Гоббса:

\begin{enumerate}
    \item Тело
    \item Движение тела в пространстве
\end{enumerate}

Основа материальных тел — Бог.

Процесс познания: объект вызывает у субъекта ощущение, которое представляет собой некое движение; движение же при взаимодействии с чувственной природой субъекта порождает реакцию на движение, то есть новое (ответное) движение.

\subsubsection{Теория общественного договора}

Теория общественного договора — \textbf{концепция происхождения государства}.

Суть — люди договорились принять соглашение о превращении своего \textbf{«естественного состояния»} в \textbf{«состояние гражданское»}. 

\hfill

Т. Гоббс  в своем труде «Левиафан» указывал на то, что люди, находясь в своем «естественном состоянии», прибывали в положении «войны всех против всех».

Естественное состояние общества Гоббс определил как \textbf{исключительно индивидуалистическое}, поскольку в  таком  состоянии люди  равны  между  собой,  свободны  и  независимы  друг от  друга, а  по  своей естественной природе  к  тому  же являются  существами  эгоистичными,  всегда  действующими  из соображений собственной выгоды и безопасности.

И для того, чтобы обезопасить себя, они решили заключить договор, по которому \textbf{отказываются от полной индивидуальной свободы в пользу государства}, обеспечивающего социальный порядок.

Задачами такого государства должны быть \textbf{безопасность, стабильность и процветание} его народа.

\subsection{Ф. Бэкон: метод индукции, учение об идолах}

\subsubsection{Метод индукции}

\paragraph{Введение} Индуктивный метод познания у Бэкона значительно отличался от того, коим пользовались ранее. В целом, суть была одна — переход от частного к общему, но важно было само понимание этих двух, путь от одного к другому.

Индукция должна была разделять природу посредством \textbf{разграничений и исключений}. И, после достаточного количества отрицательных суждений, заключать о положительном. Огромное значение уделялось при этом \textbf{аксиомам — законам, формам отличия вещей}. Бэкон утверждал, что следует переходить от частностей к меньшим аксиомам, а затем к средним и, наконец, к общим.

Основное отличие такой индукции от предшествующей — \textbf{сосредоточенность на опыте и частных случаях, выведение аксиом}, строго исходя только из них, тогда как предыдущая индукция лишь бегло их касалась, строя гипотезы практически из ничего.

\paragraph{Аксиомы}

Бэкон считал, что правильно открытые и установленные \textbf{аксиомы} влекут за собой многочисленные ряды практических приложений. Эти аксиомы открывают исследователю \textbf{формы или истинные отличия вещей}.

Для наук следует ожидать пользы тогда, когда исследование восходит по непрерывным ступеням — от частностей к меньшим аксиомам и затем к средним, одна выше другой, и наконец к самым общим.

\paragraph{Формы}

По Бэкону, в природе не существует ничего действительно, помимо \textbf{единичных тел, осуществляющих сообразно с законом отдельные чистые действия}.

Этот закон и его подразделения Бэкон называет \textbf{формами}.

\paragraph{Исследование форм}

Бэкон неоднократно предупреждает, что никто успешно не отыщет природу вещи в самой вещи — исследование \textbf{должно быть расширено до более общего}, поскольку то, что в одних вещах считается скрытым, в других имеет явную и обычную природу.

Исследование форм происходит следующим образом:

\begin{enumerate}
    \item Сначала нужно для каждой данной природы представить в таблице все известные примеры, сходящиеся в этой природе, хотя бы и посредством самых различных материй. Эта таблица называется таблицей присутствия
    \item Во-вторых, должно представить разуму примеры, которые лишены данной природы (в предметах наиболее родственных тем, в которых данная природа присутствует), так как форма так же должна отсутствовать там, где отсутствует природа, как и присутствовать там, где она присутствует. Это таблица отсутствия в ближайшем
    \item В-третьих, должно представить разуму примеры, в которых исследуемая природа присутствует в большей и в меньшей степени. Это возможно или посредством сопоставления роста и уменьшения этого свойства в одном и том же предмете, или посредством сравнения его в различных предметах. Это таблица степеней
\end{enumerate}

Эти три таблицы представляют исследователю примеры. За этим следует \textbf{сама индукция}:

\begin{displayquote}
Первое дело истинной индукции есть отбрасывание отдельных природ, которые не встречаются в каком-либо примере, где присутствует данная природа, или встречаются в каком-либо примере, где отсутствует данная природа, или встречаются растущими в каком-либо примере, где данная природа убывает, или убывают, когда данная природа растёт. После сделанного должным образом исключения останется положительная и хорошо определённая форма.
\end{displayquote}

\subsubsection{Учение об идолах}

\textbf{Идолы}, согласно Фрэнсису Бэкону — причины, которые \textbf{препятствуют человеку и человечеству получить истинное знание}:

\begin{enumerate}
    \item Врожденные заблуждения:
    \begin{enumerate}
        \item Идолы рода — преломление познания через культуру человека (рода в целом) — человек осуществляет познания, находясь в рамках общечеловеческой культуры, и это снижает истинность знания
        \item Идолы пещеры — влияние личности конкретного человека на процесс познания — личность человека (его предрассудки, заблуждения — «пещера») отражается в конечном результате познания
    \end{enumerate}
    \item Приобретенные заблуждения:
    \begin{enumerate}
        \item Идолы рынка — неправильное, неточное употребление понятийного аппарата: слов, дефиниций
        \item Идолы театра — влияние существующей философии на процесс познания — старая философия мешает проявлять новаторский подход, направляет познание не всегда в нужное русло
    \end{enumerate}
\end{enumerate}

\subsection{Д. Локк: «tabula rasa», естественные права человека}

\subsubsection{tabula rasa}

В философии Локка tabula rasa была теорией, согласно которой \textbf{при рождении (человеческий) разум представляет собой "чистый лист"} без правил обработки данных, и что данные добавляются, а правила обработки формируются исключительно \textbf{на основе чувственного опыта}.

\hfill

В понимании Локка tabula rasa означала, что \textbf{ум человека рождается пустым}, а также подчеркивает свободу людей создавать свою собственную душу. Индивиды \textbf{свободны определять содержание своего характера}, но \textbf{основная идентичность как члена человеческого вида не может быть изменена}.

\subsubsection{Естественные права человека}

Локк полагал, что еще до того, как появилось государство, \textbf{человеку природой были даны естественные права: на жизнь, свободу и собственность} — часть человека. Идея состояла в том, что люди не рождаются королями или рабами, — они \textbf{рождаются свободными, с правом распоряжаться самостоятельно своей жизнью и собственностью}.

\hfill

Человек \textbf{не может бросить эти права или передать другому}, также как и \textbf{не может их у кого-то отобрать}. Тому есть две причины: во-первых, по мнению Локка, \textbf{человек находится в собственности у Бога}, который его создал; во-вторых, свобода не дает человеку право делать все, что угодно.

\hfill

Государство было создано для того, чтобы \textbf{урегулировать права людей}, данные природой — \textbf{человек не может не быть частью социума} и на подсознательном уровне \textbf{дает свое внутреннее согласие} на такие условия.

\subsection{Рационализм в философии нового времени: общая характеристика, основные черты}

Эмпиризм в какой-то мере стал стимулом к развитию такого направления философии как рационализм. В отличие от эмпиризма, рационализм объявлял \textbf{источником знания разум}, а \textbf{основным методом познания – дедукцию}.

Основатель рационализма — Рене Декарт (1596-1650) — заявлял, что восприятие может обмануть, а полученные в ходе эксперимента данные не всегда прямо указывают на причину изучаемого явления.

Поэтому \textbf{любое наблюдение, предположение, знание должны пройти проверку разумом}, в ходе которой следует \textbf{отбросить все авторитеты и подвергнуть сомнению все}, даже кажущиеся незыблемыми истины.

\hfill

\textbf{Но} эмпиризм и рационализм в философии нельзя считать строго противоположными течениями. Любой эксперимент подлежит осмыслению, методы индукции и дедукции абсолютно равноправны и могут применяться в рамках даже одного исследования для проверки выводов и поиска неточностей.

\subsection{Р. Декарт: учение о методе, «cogito ergo sum»}

\subsubsection{Учение о методе}

Рационализм Декарта основывается на идеи \textbf{всеобщей математизации научного познания}

\paragraph{Основные положения}

Суть метода Декарта сводится к \textbf{двум положениям}:

\begin{enumerate}
    \item В познании следует отталкиваться от \textbf{интеллектуальной интуиции}, рождающейся в здравом уме, знания настолько простого и отчетливого, что оно \textbf{не вызывает никакого сомнения}
    \item Разум должен из этих интуитивных воззрений \textbf{на основе дедукции} вывести все необходимые следствия
\end{enumerate}

\paragraph{Дедукция}

Декарт сформулировал следующие \textbf{3 правила} дедуктивного метода:

\begin{enumerate}
    \item Во всяком вопросе \textbf{должно содержаться неизвестное}
    \item Это неизвестное \textbf{должно иметь какие-то характерные особенности}, чтобы исследование было направлено на постижение именно этого неизвестного
    \item В вопросе также \textbf{должно содержаться нечто известное}
\end{enumerate}

\paragraph{Правила познания истины}

\begin{enumerate}
    \item Истинно то, что очевидно и отчетливо. Постичь истину до конца можно только интеллектуальной интуицией
    \item Сложные идеи нужно раскладывать на простые
    \item Идти от простых истин к сложным
    \item Расчлененные простые истины связывать с помощью интуиции
\end{enumerate}

\paragraph{Классы идей}

\begin{enumerate}
    \item Потусторонние — появляющиеся извне
    \item Фантазии — образуемые человеком
    \item От Бога — врожденные идеи
\end{enumerate}

\paragraph{Бытие}

Бытие образует две субстанции:

\begin{enumerate}
    \item Материальную, ее атрибут — протяженность
    \item Духовную, ее атрибут — мышление
\end{enumerate}

Интеллектуальная интуиция у Декарта \textbf{начинается с сомнения} — помогает \textbf{избавиться от предрассудков} (идолов по Бэкону).

\hfill

Поставив под сомнение достоверность всех наших представлений о мире, мы можем допустить, писал Декарт, что положение «Я мыслю, следовательно, я существую» и есть представление о том, что \textbf{мышление, независимо от его содержания и объектов, демонстрирует реальность мыслящего субъекта} и \textbf{является той первичной исходной интеллектуальной интуицией}, из которой выводятся все знания о мире

\hfill

Принцип ясности и отчетливости знания гарантирован у Декарта существованием \textbf{Бога — совершенного и всемогущего, вложившего в человека свет разума}, врожденные идеи — к ним Декарт относил идею Бога как существа всесовершенного, идеи чисел и фигур.

\hfill

Согласно Декарту, \textbf{источником заблуждений не может быть разум} сам по себе. Заблуждения есть продукт \textbf{злоупотребления человеком присущей ему свободой воли}. Заблуждения возникают тогда, когда бесконечно свободная воля переступает границы конечного человеческого разума.

Но Декарт верит в \textbf{неограниченные возможности человеческого разума в деле познания} всей окружающей его действительности.

\subsubsection{Cogito ergo sum}

Cogito ergo sum — «я мыслю, следовательно, я существую» — философское утверждение Рене Декарта, \textbf{фундаментальный элемент западного рационализма} Нового времени.

\hfill

Это утверждение Декарт выдвинул как \textbf{первичную достоверность, истину, в которой невозможно усомниться} — и с которой, следовательно, можно \textbf{начинать отстраивать здание достоверного знания}.

\hfill

Аргумент \textbf{не следует понимать как умозаключение}; напротив, его \textbf{суть — в очевидности}, самодостоверности моего существования как мыслящего субъекта: всякий акт мышления (и шире — всякое представление, переживание сознания, ибо cogito не ограничивается мышлением) обнаруживает — при рефлексивном взгляде на него — меня, мыслящего, осуществляющего этот акт. \textbf{Аргумент указывает на самообнаружение субъекта в акте мышления} (сознания): я мыслю — и, созерцая своё мышление, обнаруживаю себя, мыслящего, стоящего за его актами и содержаниями.

\end{flushleft}

\pagebreak
\section{Философия - 24.11.2022}

\subsection{Иммануил Кант}

\begin{flushleft}

Родился в 1725 году - умер в самом начале 19-го века от Альцгеймера. Жил в Konigsbergе.

Изобретатель \textbf{космополитизма} - был уверен в том, что рано или поздно на планете наступет вечный мир - трактат ,К вечному миру'. Причина разногласий между людьми - существование различных государств. Решение - \textbf{общемировое государство} 

\subsubsection{Докритический период}

Узкоспециальная тема, не такая интересная.

Докритический период - период, когда Кант занимался \textbf{естественными науками}.

Самое известное произведение периода: \textbf{"Естественная история неба"}.

\hfill

«Грёзы духовидца, пояснённые грёзами метафизики» - допереходный период.

\hfill

Переходный период - десять лет молчания. Посвятил работе над главным трудом свое жизни - \textbf{«Критика чистого разума»}

\subsubsection{Критический период}

\textbf{«Критика чистого разума»} представляет собой некую черту в истории развития философской мысли. Критикует разум, говорит о границах и возможностях человеческого разума.

Кант обозначает целью и задачей своей работы ответить на вопрос о том, \textbf{как возможны синтетические суждения априори} (априори - букв. доопытный)

Кант разделяет суждения на две категории:

\begin{enumerate}
    \item Синтетические - предикат возможен без субъекта, прибавляется к субъекту извне
    \item Аналитические - предикат исходит из субъекта
\end{enumerate}

\textbf{Априорные синтетические суждения} - без всякого опыта можем прибавить что-либо к субъекту высказывания.

\paragraph{Гносеология} \textbf{Трансцендентальное единство апперцепции}. Выделяет три уровня в познавательном аппарате:

\begin{enumerate}
    \item \textbf{Чувственность} - то, что поставляет нам \textbf{,чистые созерцания'}. У нашей чувственности есть условия - \textbf{априорные формы чувственности} - пространство (априорная форма внешнего чувства) и время (априорная форма внутреннего чувства) - трансцендентальная трактовка (\textbf{трансцендентальный} - внутренние условия, благодаря которым возможен опыт). Чистые созерцания \textbf{не являются опытом}. Пространство и время помещены внутрь чувственности. Смена фокуса философии с \textbf{объективного} к \textbf{субъективному}
    \item \textbf{Рассудок} - двенадцать категорий чистого рассудка - четыре группы - \textbf{количество}, \textbf{качество}, \textbf{отношение} и \textbf{модальность}. Не имеют смысла без объекта. \textbf{Опыт} - \textbf{сочетание чистого созерцания и чистого рассудка}. \textbf{Логическое} различие между чувственностью и рассудком, онтологического различия не имеется. В нашем разуме есть особые способы связывать чувственность и рассудок - \textbf{схематизмы чистого рассудка} - схемы, по которым соединяются понятия и созерцания - связующее звено между рассудком и чувственностью - применяем с помощью способности к воображению - получаем опыт.
    \item \textbf{Разум}. То место, где находится трансцендентальный субъект - условие возможности познания - тот, кто осуществляет познание.
\end{enumerate}

Мы конструируем окружающую нас реальность - создаем ее у нас в уме - все знание мы производим сами. Разделение мира на мир \textbf{феноменальный} (то, как мир нам является) и \textbf{ноуменальный} (мир вещей самих по себе - то, какова есть вещь на самом деле; мы \textbf{не способны его познать}). Все, что мы можем сказать о себе сами - феномены, не имеющиеся никакие отношения к нам самих по себе.

\paragraph{Этика} «Критика практического разума». Этика - практическая философия.

\textbf{Императивная этика} - содержит указания, как надо поступать.

Этика Канта - этика долга. Наш долг - в следовании \textbf{нравственному закону} - \textbf{категорическому императиву}.

«Поступай так, чтобы максима твоей воли смогла стать основой всеобщего законодательства» - \textbf{поступай так, как ты бы хотел, чтобы поступали все}.

Выступал критиком золотого правила этики - в основе своей \textbf{аморально} - строго эгоистично. Согласно Канту, нравственное поведение всегда бескорыстное.

\textbf{Всегда относись к другому человеку только как к цели, но никогда как к средству}

\hfill

Перед тем, как решится на какое-то действие, согласно Канту, должно \textbf{представить мир, в котором поступают все так, как вы собираетесь поступить}. Если такой мир возможен - ваш поступок является нравственным, если нет - безнравственным.

Категорический императив \textbf{применим не всегда}.

Кант отмечает, что следование категорическому императиву это \textbf{всегда свободный выбор человека}.

Свобода - \textbf{центральная категория кантианской этики}

\textbf{Поступлаты практического разума} - необходимые условия нравственности - без них нравственность является невозможной:

\begin{enumerate}
    \item Свобода воли - только \textbf{свободный поступок} может быть нравственным.
    \item Бессмертие души - мир несправедлив и злым быть выгодно - вера в то, что после нашей смерти будет установлена и восстановлена справедливость.
    \item Существование бога (\textbf{как идеи нашего разума}) - гарант справедливости - вознаградит тех, кто был молодец, и накажет тех, кто не был молодец.
\end{enumerate}

\hfill

\textbf{Связь между теоретическим и практическим разумом}

\textbf{Противоречие} между гносеологией и свободой воли - мое моральное поведение \textbf{не более, чем конструкция моего разума} - свобода воли оказывается запертой в границах разума.

Свобода может быть только тем, что мы знаем, и ничем другим. Свобода оказывается \textbf{несвободной} - невозможной - этика оказывается невозможной.

Парадокс является \textbf{выражением абсолютной свободы} - может быть чем угодно - Кант сознательно его оставляет, спасает нашу возможность быть нравственными.

\paragraph{Критика способности суждения} Еще одна критика - эстетика и телеология - \textbf{понятие целесообразности}.

\paragraph{Просвещение} Просвещение - \textbf{публичное использование разума}

\end{flushleft}

\pagebreak
\section{Философия - 24.11.2022}

\subsection{Славянофильство и западничество}

\subsubsection{Славянофильство}

\begin{flushleft}

Славянофильство — литературное и религиозно-философское течение русской общественной и философской мысли, оформившееся в 30-х—40-х годах XIX века и ориентированное на выявление \textbf{самобытности России}, её типовых отличий от Запада.

Представители выступали за развитие \textbf{особого русского пути, отличного от западноевропейского}. Развиваясь по нему, по их мнению, Россия способна донести православную истину до впавших в ересь и атеизм европейских народов.

Славянофилы утверждали также о существовании \textbf{особого типа культуры}, возникшего на духовной почве православия, а также отвергали тезис представителей западничества о том, что Пётр I возвратил Россию в лоно европейских стран, и она должна пройти этот путь в политическом, экономическом и культурном развитии.

\paragraph{Представители} Основоположником славянофильства стал литератор А. С. Хомяков, деятельную роль в движении играли И. В. Киреевский, К. С. Аксаков, И. С. Аксаков, Ю. Ф. Самарин, А. И. Кошелев, Ф. В. Чижов.

Умеренные славянофилы — А. А. Григорьев, Н. Н. Страхов, Н. Я. Данилевский, К. Н. Леонтьев, Ф. М. Достоевский, Ф. И. Тютчев, В. И. Даль

\subsubsection{Западничество}

Западничество — сложившееся в 1830—1850-х годах направление общественной и философской мысли.

Западники, представители одного из направлений русской общественной мысли 40—50-х годов XIX века, выступали за \textbf{отмену крепостного права} и признание необходимости \textbf{развития России по западноевропейскому пути}. Большинство западников по происхождению и положению принадлежали к дворянам-помещикам, были среди них разночинцы и выходцы из среды богатого купечества, ставшие впоследствии преимущественно учёными и писателями.

\paragraph{Представители}

Идеи западничества выражали и пропагандировали публицисты и литераторы — П. Я. Чаадаев, В. С. Печерин, И. А. Гагарин, В. С. Соловьёв (представители так называемого \textbf{религиозного западничества}), И. С. Тургенев и Б. Н. Чичерин (\textbf{либеральные западники}), В. Г. Белинский, А. И. Герцен, Н. П. Огарёв, позднее Н. Г. Чернышевский, В. П. Боткин, П. В. Анненков (\textbf{западники-социалисты}), М. Н. Катков, Е. Ф. Корш, А. В. Никитенко и др.; \textbf{профессора истории, права и политической экономии} — Т. Н. Грановский, П. Н. Кудрявцев, С. М. Соловьёв, К. Д. Кавелин, Б. Н. Чичерин, П. Г. Редкин, И. К. Бабст, И. В. Вернадский и др. Идеи западников в той или иной степени разделяли \textbf{писатели, поэты, публицисты} — Н. А. Мельгунов, Д. В. Григорович, И. А. Гончаров, А. В. Дружинин, А. П. Заблоцкий-Десятовский, В. Н. Майков, В. А. Милютин, Н. А. Некрасов, И. И. Панаев, А. Ф. Писемский, М. Е. Салтыков-Щедрин, но они часто пытались примирить западников и славянофилов, хотя с годами в их взглядах и творчестве прозападническое направление преобладало.

\subsection{Русская религиозная философия: В.С. Соловьев, Н.А. Бердяев}

 \subsubsection{Владимир Сергеевич Соловьёв}

Стержень философии — \textbf{категория «всеединства»}

\paragraph{Всеединство}

Идея всеединства выражает \textbf{органическое единство мирового бытия}, наличие взаимопроникновения составляющих его элементов при \textbf{сохранении их индивидуальности}.

Всеединство представляет собой \textbf{нерасторжимое единение творца и твари}; «цельное знание», представляющее \textbf{неразрывную взаимосвязть} эмпирического, рационального и мистического знания, достигаемого не только и не столько в результате познавательной деятельности, сколько \textbf{верой и интуицией}.

Центральное место занимает \textbf{абсолютная ценность истины, добра и красоты}, соответствующих трем ипостасям Божественной Троицы.

Сам Владимир Соловьев определял всеединство следующим образом:  «Я называю истинным, или положительным, всеединством такое, в котором единое существует \textbf{не за счёт всех или в ущерб им, а в пользу всех} … истинное единство \textbf{сохраняет и усиливает свои элементы}, осуществляясь в них как полнота бытия»

\paragraph{София}

Основная идея философии — \textbf{София — Душа Мира}, понимаемая как мистическое космическое существо, \textbf{объединяющее Бога с земным миром}.

София представляет собой \textbf{вечную женственность в Боге} и, одновременно, \textbf{замысел Бога о мире}.

Идея Софии реализуется трояким способом: в \textbf{теософии} формируется представление о ней, в \textbf{теургии} она обретается, а в \textbf{теократии} она воплощается.

\begin{enumerate}
    \item Теософия — дословно \textbf{божественная мудрость} — представляет собой \textbf{синтез научных открытий и откровений христианской религии} в рамках цельного знания. Соловьёв признает идею эволюции, но считает её попыткой преодоления грехопадения через прорыв к Богу. Эволюция проходит через \textbf{пять этапов}: минеральное, растительное, животное, человеческое и Божье.
    \item Теургия — дословно \textbf{боготворчество} — \textbf{очистительная практика}, без которой невозможно обретение истины — \textbf{культивирование христианской любви} как \textbf{отречение от самоутверждения} ради единства с другими.
    \item Теократия — дословно \textbf{власть Бога} — заключается в «\textbf{истинной солидарности} всех наций и классов», а также в «христианстве, осуществлённом в общественной жизни» — Владимир Соловьёв \textbf{возлагал «теократическую миссию» на Россию}
\end{enumerate}

\subsubsection{Николай Александрович Бердяев}

Бердяев начал свою философскую деятельность как \textbf{марксист} — пафос революции, критика буржуазности; но потом он все более склонялся к \textbf{философии экзистенциализма и персонализма}. Критика отождествления философии с наукой. Научность — «рабство духа у низших сфер бытия». Философия — \textbf{искусство} — важная роль творчества, личности и призвания.

Идеолог персонализма — верил в \textbf{исключительность каждой личности} и её силу — внутренняя жизнь \textbf{отчуждена от внешнего мира} — видел конфликт с внешним миром и обществом, стремящимся подавить его внутреннюю духовную жизнь.

\paragraph{Историческая память}

\textbf{Историческая память}  — два уровня: первый — связана с самой категорией времени, а также с богом; второй — связана с непосредственно конкретными историческими событиями из жизни народа.

\paragraph{Отношение к христианству}

Бердяев благосклонно относится к христианству — «знаком образа Творца» в человеке является «творческая свобода». Видел \textbf{сходство всех религий в идее преодоления мира}, поэтому он вводил понятие «\textbf{нового религиозного сознания}».


\paragraph{Основные понятия философии}

Основными понятиями философии Бердяева является \textbf{свобода} (как антитеза необходимости), в которой творчески преодолевается власть отчуждения. Учение о \textbf{«первичной», «несотворённой» свободе}, над которой не властен даже Бог. Он противопоставляет «свободу от» (свободу в негативном смысле) «свободе для». Христианство — \textbf{религия свободы} — в нём появляется возможность преодоления внешних обстоятельств с помощью действий свободного субъекта.

Культура — объективация и побочный результат творчества, которое стремится к преображению мира — не локализует творчество в одном Боге, но выдвигает учение о «\textbf{восьмом дне творения}», то есть продолжающемся творении, в котором участвует и человек.

\subsection{Русская социальная философия: А.И. Герцен, В.И. Ленин}

\subsubsection{Александр Иванович Герцен}

\textbf{Синтезировал западничество и славянофильство}, в первом отринув либерализм, а во втором — консерватизм и великодержавный шовинизм — новое учение, построенное на идеях \textbf{прогрессизма, гуманизма и значения личности}, с одной стороны, а с другой — \textbf{солидарности и общинности}. Свои взгляды Герцен назвал \textbf{русским социализмом}, который от крестьянской общины идёт к антигосударственным социалистическим идеалам справедливости. Чтобы достичь данной идиллии, нужно было не только \textbf{преодолеть самодержавие}, но и \textbf{просвещать крестьян} для достижения ими эмансипации, \textbf{изжить рабские установки} крепостного права.

\hfill

\textbf{Развитие человечества идёт ступенями}, и каждая ступень воплощается в известном народе. Верил в грядущую смену германского периода славянским — соединял эту веру в славянский фазис прогресса с учением о предстоящей \textbf{замене господства буржуазии торжеством рабочего класса}. Герцен разочаровывался в западной культуре, считая, что «\textbf{запад сгнил}, и в его обветшавшие формы не влить уже новой жизни». При этом Герцен не отрицал возможности того, что и Россия пройдёт через стадию буржуазного развития. Герцен был убеждён, что \textbf{славянский мир стремится к единству}, и так как «централизация противна славянскому духу», то славянство объединится на принципах федераций.

\subsubsection{Владимир Ильич Ленин}

\subsection{Русский космизм: В.И. Вернадский, Н.Ф. Федоров, К.Э. Циолковский}

\subsubsection{Владимир Иванович Вернадский}

\subsubsection{Николай Федорович Федоров}

\subsubsection{Константин Эдуардович Циолковский}

\end{flushleft}

\pagebreak
\section{Философия - 29.11.2022}

\subsection{Философия А. Шопенгауэра: философский пессимизм, воля и представление}

\subsubsection{Философский пессимизм}

\begin{flushleft}

Вся человеческая \textbf{жизнь — сплошные страдания и разочарования}. Желания никогда не удовлетворяются окончательно. Вскоре после достижения цели наступает равнодушие и скука. Между страданиями и скукой мечется человеческая жизнь.

Земное \textbf{счастье — иллюзия}. Счастье всегда находится в будущем (которое ненадежно) или в прошлом (которое уже невозможно), так что в настоящем человек никогда счастлив не бывает.

Мы не знаем трех высших благ жизни — здоровья, молодости и свободы. Пока они у нас есть, мы их не осознаем и не понимаем их ценности, а понимаем только тогда, когда утратим. Счастливые дни мы замечаем только тогда, когда они уступают место несчастным дням.

Все вокруг суетятся — одни в мечтах, другие в деятельности. Но последняя цель всего этого оказывается жалкой: поддержать на короткий промежуток времени жизнь.

Как бы ни была жалка жизнь, человек цепляется за нее из последних сил, воля к жизни сильнее разума, сам разум — продукт воли. Но эта \textbf{короткая отсрочка смерти имеет ничтожное значение — победа смерти несомненна}.

Поэтому человек должен выйти из-под власти воли, \textbf{подавить в себе всякие желания}: \textbf{понять, что страдания — неизбежная часть нашей жизни}, и если мы избавимся от одного, то неизбежно придет другое; если на время прекращаются страдания, то наступает скука, которая также является страданием. Если мы поймем это, нам удастся воспитать в себе равнодушие к страданиям. Ведь страдания, как и счастье, приходят не извне, а возникают изнутри человека. Подавить волю, перестать быть ее рабом, уменьшить тягостную заботу о собственном благополучии — таков единственно возможный путь мыслящего человека. Доступный, правда, немногим, кому дано понять, что \textbf{жизнь — это вечный обман и вечные разочарования}, что \textbf{в мире нет ничего достойного наших желаний, стремлений и борьбы, что все его блага ничтожны}. Тем самым мы преодолеваем господство воли.

\subsubsection{Учение о воле}

Шопенгауэр использовал слово \textbf{«воля»} как наиболее известное указание на концепцию, которую можно обозначить также словами \textbf{«вожделение», «стремление», «желание», «усилие», «призыв»}.

Вся природа, включая человека, - \textbf{это выражение ненасытной «воли к жизни»}.

Именно благодаря этому «желанию жить» человечество испытывает страдание. Желание большего — причина ещё больших страданий.

\subsubsection{Учение о представлении}

\textbf{Представление} - производимая в уме идея, образ или любой объект пережитого, в значении \textbf{находящегося снаружи по отношению к разуму}.

\subsection{Философия Ф. Ницше: учение о воле}

\subsubsection{Учение о воле}

\textbf{Воля к власти} - это одна из разновидностей волевых импульсов человеческого поведения. Волю к власти Ницше считал \textbf{определяющим стимулом деятельности и главной способностью человека}. Основой жизни, по Ницше, является \textbf{воля к власти или тяга всего живого к самоутверждению, всемогуществу, стремлению расширить власть}. На протяжении всей жизни человек стремится достичь максимума чувства власти.

\textbf{Простые люди никчемны, слабы, половинчаты, мягкотелы}, не способны созидать и властвовать. Они – \textbf{рабы от природы} и могут лишь подчиняться. Надо любить не слабого и ближнего, а сильного

Простому человеку у Ницше противостоит сверхчеловек – существо высшего биотипа, не принадлежащее ни к одной расе и выращенное элитой общества. Сверхчеловек – это выражение полноты жизни, создатель новых ценностей. Он находится \textbf{вне всяких моральных норм, вне добра и зла, с особой жестокостью преодолевает тотальную ложь земного мира}.

Неоевропейскому рационализму Ницше противопоставил \textbf{чувства и инстинкты}, обеспечивающие, с его точки зрения, волю к власти. Он полагал, что разум по своей сути ничтожен, логика абсурдна, так как имеет дело с застывшими формами, противоречащими динамике жизни.

Главной целью познания Ницше считал \textbf{овладение миром}, а не установление истины, которая тождественна заблуждению. Заблуждение и ложь необходимы, поскольку утешают толпу и помогают аристократам духа самоутверждаться.

\subsection{Философия А. Бергсона: жизненный порыв, деятельность}

\subsubsection{Учение о жизненном порыве}

\textbf{Жизненный порыв - метафора}, с помощью которой Бергсон сформулировал ряд важных для него \textbf{эволюционных идей}.

Жизнь \textbf{зарождается в одном центре} в силу начального импульса, а затем ее порыв продвигается по \textbf{множеству параллельных направлений}, претерпевая по пути целую серию качественных скачков, подобных взрывам. Из-за сопротивления, оказываемого материей, на большинстве линий порыв угасает и развитие прекращается, сменяясь круговоротом. 

\textbf{Среди направлений движения} порыва Бергсон \textbf{выделяет три основных – чувственность, инстинкт и интеллект}, которые приводят, соответственно, \textbf{к растениям, животным и человеку}.

\subsubsection{Учение о длительности}

Бергсон стремился устранить недостатки, которые он увидел в философии Герберта Спенсера, в связи, как он считал, с недостатком знаний Спенсера о механике, что привело Бергсона к выводу, что время ускользает от математики и естественных наук. Бергсон понял, что момент, когда человек пытался измерить время, ушёл: измеряется неподвижная, полная линия, тогда как время подвижное и неполное. В отдельных случаях, время может ускоряться или замедляться, тогда как для науки оно останется прежним. Таким образом, Бергсон решил исследовать внутренний мир человека, который является типом продолжительности, ни единство и не количественную множественность. Длительность невыразима и может быть отображена только косвенно, через образы, которые никогда не могут показать полную картину. Это может быть постигнуто только с помощью интуиции воображения.

Бергсон представляет три изображения длительности. Первое имеет две катушки: одна разворачивающаяся, чтобы представлять непрерывный поток старения, чувствуя своё приближение к концу продолжительности жизни, другая - сворачивающаяся, чтобы показать непрерывный рост памяти, которая, по мнению Бергсона, равняется сознанию. У человека без памяти могут возникнуть два одинаковых момента, но, говорит Бергсон, осознавая, что человек, таким образом, будет находиться в состояние смерти и возрождения, которую он отождествляет с потерей сознания. Изображение двух катушек, несмотря на то, что они из однородной и пропорциональной нити, в то время, как Бергсон считал, что никакие два момента не могут быть одинаковыми, поэтому продолжительность неоднородна.

Бергсон потом предоставил изображение спектра тысячи постепенно меняющихся оттенков с линией, которая проходит через них, находясь под влиянием и поддерживая каждый оттенок. Но даже этот образ является неточным и неполным, поскольку он представляет длительность в виде фиксированного и полного спектра всех оттенков, сопоставленных в пространстве, в то время как продолжительность неполная и постоянно растёт, её состояние это не начало и не конец, а что-то смешанное.

На самом деле, мы меняемся, не переставая… нет существенной разницы между переходом от одного состояния к другому и сохранения в том же состоянии. Если состояние, которое «остаётся неизменным» является более разнообразным, чем мы думаем, то с другой стороны, переход из одного состояния в другое напоминает больше, чем мы представляли — единое состояние продлевается: переход является непрерывным. Только потому, что мы закрываем наши глаза на непрерывное изменение каждого физического состояния, мы вынуждены, когда изменения стали настолько грозными, чтобы привлечь наше внимание, говорить, как если бы новое состояние было бы размещено рядом с предыдущими. Мы считаем, что это новое состояние, в свою очередь, остаётся неизменным и так до бесконечности.

Потому, что качественная кратность неоднородна и ещё проникающая себя, она не может быть адекватно представлена символом, да и для Бергсона, качественная кратность невыразима. Таким образом, чтобы понять длительность, нужно отойти от привычных способов мышления и поставить себя в течение длительности с помощью интуиции.

\subsection{Рекомендуемая литература}

\begin{enumerate}
    \item Кротов А. А., Бугай Д. В. История философии. М.: Академический проспект, 2008
\end{enumerate}

\end{flushleft}

\end{document}