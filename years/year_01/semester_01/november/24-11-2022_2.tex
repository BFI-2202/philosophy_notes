\documentclass{article}
\usepackage[utf8]{inputenc}

\usepackage[T2A]{fontenc}
\usepackage[utf8]{inputenc}
\usepackage[polutonikogreek, english,russian]{babel}
\usepackage{csquotes}

\title{Философия}
\author{Лисид Лаконский}
\date{November 2022}

\begin{document}

\maketitle
\tableofcontents
\pagebreak

\section{Философия - 24.11.2022}

\subsection{Славянофильство и западничество}

\subsubsection{Славянофильство}

\begin{flushleft}

Славянофильство — литературное и религиозно-философское течение русской общественной и философской мысли, оформившееся в 30-х—40-х годах XIX века и ориентированное на выявление \textbf{самобытности России}, её типовых отличий от Запада.

Представители выступали за развитие \textbf{особого русского пути, отличного от западноевропейского}. Развиваясь по нему, по их мнению, Россия способна донести православную истину до впавших в ересь и атеизм европейских народов.

Славянофилы утверждали также о существовании \textbf{особого типа культуры}, возникшего на духовной почве православия, а также отвергали тезис представителей западничества о том, что Пётр I возвратил Россию в лоно европейских стран, и она должна пройти этот путь в политическом, экономическом и культурном развитии.

\paragraph{Представители} Основоположником славянофильства стал литератор А. С. Хомяков, деятельную роль в движении играли И. В. Киреевский, К. С. Аксаков, И. С. Аксаков, Ю. Ф. Самарин, А. И. Кошелев, Ф. В. Чижов.

Умеренные славянофилы — А. А. Григорьев, Н. Н. Страхов, Н. Я. Данилевский, К. Н. Леонтьев, Ф. М. Достоевский, Ф. И. Тютчев, В. И. Даль

\subsubsection{Западничество}

Западничество — сложившееся в 1830—1850-х годах направление общественной и философской мысли.

Западники, представители одного из направлений русской общественной мысли 40—50-х годов XIX века, выступали за \textbf{отмену крепостного права} и признание необходимости \textbf{развития России по западноевропейскому пути}. Большинство западников по происхождению и положению принадлежали к дворянам-помещикам, были среди них разночинцы и выходцы из среды богатого купечества, ставшие впоследствии преимущественно учёными и писателями.

\paragraph{Представители}

Идеи западничества выражали и пропагандировали публицисты и литераторы — П. Я. Чаадаев, В. С. Печерин, И. А. Гагарин, В. С. Соловьёв (представители так называемого \textbf{религиозного западничества}), И. С. Тургенев и Б. Н. Чичерин (\textbf{либеральные западники}), В. Г. Белинский, А. И. Герцен, Н. П. Огарёв, позднее Н. Г. Чернышевский, В. П. Боткин, П. В. Анненков (\textbf{западники-социалисты}), М. Н. Катков, Е. Ф. Корш, А. В. Никитенко и др.; \textbf{профессора истории, права и политической экономии} — Т. Н. Грановский, П. Н. Кудрявцев, С. М. Соловьёв, К. Д. Кавелин, Б. Н. Чичерин, П. Г. Редкин, И. К. Бабст, И. В. Вернадский и др. Идеи западников в той или иной степени разделяли \textbf{писатели, поэты, публицисты} — Н. А. Мельгунов, Д. В. Григорович, И. А. Гончаров, А. В. Дружинин, А. П. Заблоцкий-Десятовский, В. Н. Майков, В. А. Милютин, Н. А. Некрасов, И. И. Панаев, А. Ф. Писемский, М. Е. Салтыков-Щедрин, но они часто пытались примирить западников и славянофилов, хотя с годами в их взглядах и творчестве прозападническое направление преобладало.

\subsection{Русская религиозная философия: В.С. Соловьев, Н.А. Бердяев}

 \subsubsection{Владимир Сергеевич Соловьёв}

Стержень философии — \textbf{категория «всеединства»}

\paragraph{Всеединство}

Идея всеединства выражает \textbf{органическое единство мирового бытия}, наличие взаимопроникновения составляющих его элементов при \textbf{сохранении их индивидуальности}.

Всеединство представляет собой \textbf{нерасторжимое единение творца и твари}; «цельное знание», представляющее \textbf{неразрывную взаимосвязть} эмпирического, рационального и мистического знания, достигаемого не только и не столько в результате познавательной деятельности, сколько \textbf{верой и интуицией}.

Центральное место занимает \textbf{абсолютная ценность истины, добра и красоты}, соответствующих трем ипостасям Божественной Троицы.

Сам Владимир Соловьев определял всеединство следующим образом:  «Я называю истинным, или положительным, всеединством такое, в котором единое существует \textbf{не за счёт всех или в ущерб им, а в пользу всех} … истинное единство \textbf{сохраняет и усиливает свои элементы}, осуществляясь в них как полнота бытия»

\paragraph{София}

Основная идея философии — \textbf{София — Душа Мира}, понимаемая как мистическое космическое существо, \textbf{объединяющее Бога с земным миром}.

София представляет собой \textbf{вечную женственность в Боге} и, одновременно, \textbf{замысел Бога о мире}.

Идея Софии реализуется трояким способом: в \textbf{теософии} формируется представление о ней, в \textbf{теургии} она обретается, а в \textbf{теократии} она воплощается.

\begin{enumerate}
    \item Теософия — дословно \textbf{божественная мудрость} — представляет собой \textbf{синтез научных открытий и откровений христианской религии} в рамках цельного знания. Соловьёв признает идею эволюции, но считает её попыткой преодоления грехопадения через прорыв к Богу. Эволюция проходит через \textbf{пять этапов}: минеральное, растительное, животное, человеческое и Божье.
    \item Теургия — дословно \textbf{боготворчество} — \textbf{очистительная практика}, без которой невозможно обретение истины — \textbf{культивирование христианской любви} как \textbf{отречение от самоутверждения} ради единства с другими.
    \item Теократия — дословно \textbf{власть Бога} — заключается в «\textbf{истинной солидарности} всех наций и классов», а также в «христианстве, осуществлённом в общественной жизни» — Владимир Соловьёв \textbf{возлагал «теократическую миссию» на Россию}
\end{enumerate}

\subsubsection{Николай Александрович Бердяев}

Бердяев начал свою философскую деятельность как \textbf{марксист} — пафос революции, критика буржуазности; но потом он все более склонялся к \textbf{философии экзистенциализма и персонализма}. Критика отождествления философии с наукой. Научность — «рабство духа у низших сфер бытия». Философия — \textbf{искусство} — важная роль творчества, личности и призвания.

Идеолог персонализма — верил в \textbf{исключительность каждой личности} и её силу — внутренняя жизнь \textbf{отчуждена от внешнего мира} — видел конфликт с внешним миром и обществом, стремящимся подавить его внутреннюю духовную жизнь.

\paragraph{Историческая память}

\textbf{Историческая память}  — два уровня: первый — связана с самой категорией времени, а также с богом; второй — связана с непосредственно конкретными историческими событиями из жизни народа.

\paragraph{Отношение к христианству}

Бердяев благосклонно относится к христианству — «знаком образа Творца» в человеке является «творческая свобода». Видел \textbf{сходство всех религий в идее преодоления мира}, поэтому он вводил понятие «\textbf{нового религиозного сознания}».


\paragraph{Основные понятия философии}

Основными понятиями философии Бердяева является \textbf{свобода} (как антитеза необходимости), в которой творчески преодолевается власть отчуждения. Учение о \textbf{«первичной», «несотворённой» свободе}, над которой не властен даже Бог. Он противопоставляет «свободу от» (свободу в негативном смысле) «свободе для». Христианство — \textbf{религия свободы} — в нём появляется возможность преодоления внешних обстоятельств с помощью действий свободного субъекта.

Культура — объективация и побочный результат творчества, которое стремится к преображению мира — не локализует творчество в одном Боге, но выдвигает учение о «\textbf{восьмом дне творения}», то есть продолжающемся творении, в котором участвует и человек.

\subsection{Русская социальная философия: А.И. Герцен, В.И. Ленин}

\subsubsection{Александр Иванович Герцен}

\textbf{Синтезировал западничество и славянофильство}, в первом отринув либерализм, а во втором — консерватизм и великодержавный шовинизм — новое учение, построенное на идеях \textbf{прогрессизма, гуманизма и значения личности}, с одной стороны, а с другой — \textbf{солидарности и общинности}. Свои взгляды Герцен назвал \textbf{русским социализмом}, который от крестьянской общины идёт к антигосударственным социалистическим идеалам справедливости. Чтобы достичь данной идиллии, нужно было не только \textbf{преодолеть самодержавие}, но и \textbf{просвещать крестьян} для достижения ими эмансипации, \textbf{изжить рабские установки} крепостного права.

\hfill

\textbf{Развитие человечества идёт ступенями}, и каждая ступень воплощается в известном народе. Верил в грядущую смену германского периода славянским — соединял эту веру в славянский фазис прогресса с учением о предстоящей \textbf{замене господства буржуазии торжеством рабочего класса}. Герцен разочаровывался в западной культуре, считая, что «\textbf{запад сгнил}, и в его обветшавшие формы не влить уже новой жизни». При этом Герцен не отрицал возможности того, что и Россия пройдёт через стадию буржуазного развития. Герцен был убеждён, что \textbf{славянский мир стремится к единству}, и так как «централизация противна славянскому духу», то славянство объединится на принципах федераций.

\subsubsection{Владимир Ильич Ленин}

\subsection{Русский космизм: В.И. Вернадский, Н.Ф. Федоров, К.Э. Циолковский}

\subsubsection{Владимир Иванович Вернадский}

\subsubsection{Николай Федорович Федоров}

\subsubsection{Константин Эдуардович Циолковский}

\end{flushleft}

\end{document}