\documentclass{article}
\usepackage[utf8]{inputenc}

\usepackage[T2A]{fontenc}
\usepackage[utf8]{inputenc}
\usepackage[polutonikogreek, english,russian]{babel}
\usepackage{csquotes}

\title{Философия}
\author{Лисид Лаконский}
\date{November 2022}

\begin{document}

\maketitle
\tableofcontents
\pagebreak

\section{Философия - 24.11.2022}

\subsection{Иммануил Кант}

\begin{flushleft}

Родился в 1725 году - умер в самом начале 19-го века от Альцгеймера. Жил в Konigsbergе.

Изобретатель \textbf{космополитизма} - был уверен в том, что рано или поздно на планете наступет вечный мир - трактат ,К вечному миру'. Причина разногласий между людьми - существование различных государств. Решение - \textbf{общемировое государство} 

\subsubsection{Докритический период}

Узкоспециальная тема, не такая интересная.

Докритический период - период, когда Кант занимался \textbf{естественными науками}.

Самое известное произведение периода: \textbf{"Естественная история неба"}.

\hfill

«Грёзы духовидца, пояснённые грёзами метафизики» - допереходный период.

\hfill

Переходный период - десять лет молчания. Посвятил работе над главным трудом свое жизни - \textbf{«Критика чистого разума»}

\subsubsection{Критический период}

\textbf{«Критика чистого разума»} представляет собой некую черту в истории развития философской мысли. Критикует разум, говорит о границах и возможностях человеческого разума.

Кант обозначает целью и задачей своей работы ответить на вопрос о том, \textbf{как возможны синтетические суждения априори} (априори - букв. доопытный)

Кант разделяет суждения на две категории:

\begin{enumerate}
    \item Синтетические - предикат возможен без субъекта, прибавляется к субъекту извне
    \item Аналитические - предикат исходит из субъекта
\end{enumerate}

\textbf{Априорные синтетические суждения} - без всякого опыта можем прибавить что-либо к субъекту высказывания.

\paragraph{Гносеология} \textbf{Трансцендентальное единство апперцепции}. Выделяет три уровня в познавательном аппарате:

\begin{enumerate}
    \item \textbf{Чувственность} - то, что поставляет нам \textbf{,чистые созерцания'}. У нашей чувственности есть условия - \textbf{априорные формы чувственности} - пространство (априорная форма внешнего чувства) и время (априорная форма внутреннего чувства) - трансцендентальная трактовка (\textbf{трансцендентальный} - внутренние условия, благодаря которым возможен опыт). Чистые созерцания \textbf{не являются опытом}. Пространство и время помещены внутрь чувственности. Смена фокуса философии с \textbf{объективного} к \textbf{субъективному}
    \item \textbf{Рассудок} - двенадцать категорий чистого рассудка - четыре группы - \textbf{количество}, \textbf{качество}, \textbf{отношение} и \textbf{модальность}. Не имеют смысла без объекта. \textbf{Опыт} - \textbf{сочетание чистого созерцания и чистого рассудка}. \textbf{Логическое} различие между чувственностью и рассудком, онтологического различия не имеется. В нашем разуме есть особые способы связывать чувственность и рассудок - \textbf{схематизмы чистого рассудка} - схемы, по которым соединяются понятия и созерцания - связующее звено между рассудком и чувственностью - применяем с помощью способности к воображению - получаем опыт.
    \item \textbf{Разум}. То место, где находится трансцендентальный субъект - условие возможности познания - тот, кто осуществляет познание.
\end{enumerate}

Мы конструируем окружающую нас реальность - создаем ее у нас в уме - все знание мы производим сами. Разделение мира на мир \textbf{феноменальный} (то, как мир нам является) и \textbf{ноуменальный} (мир вещей самих по себе - то, какова есть вещь на самом деле; мы \textbf{не способны его познать}). Все, что мы можем сказать о себе сами - феномены, не имеющиеся никакие отношения к нам самих по себе.

\paragraph{Этика} «Критика практического разума». Этика - практическая философия.

\textbf{Императивная этика} - содержит указания, как надо поступать.

Этика Канта - этика долга. Наш долг - в следовании \textbf{нравственному закону} - \textbf{категорическому императиву}.

«Поступай так, чтобы максима твоей воли смогла стать основой всеобщего законодательства» - \textbf{поступай так, как ты бы хотел, чтобы поступали все}.

Выступал критиком золотого правила этики - в основе своей \textbf{аморально} - строго эгоистично. Согласно Канту, нравственное поведение всегда бескорыстное.

\textbf{Всегда относись к другому человеку только как к цели, но никогда как к средству}

\hfill

Перед тем, как решится на какое-то действие, согласно Канту, должно \textbf{представить мир, в котором поступают все так, как вы собираетесь поступить}. Если такой мир возможен - ваш поступок является нравственным, если нет - безнравственным.

Категорический императив \textbf{применим не всегда}.

Кант отмечает, что следование категорическому императиву это \textbf{всегда свободный выбор человека}.

Свобода - \textbf{центральная категория кантианской этики}

\textbf{Поступлаты практического разума} - необходимые условия нравственности - без них нравственность является невозможной:

\begin{enumerate}
    \item Свобода воли - только \textbf{свободный поступок} может быть нравственным.
    \item Бессмертие души - мир несправедлив и злым быть выгодно - вера в то, что после нашей смерти будет установлена и восстановлена справедливость.
    \item Существование бога (\textbf{как идеи нашего разума}) - гарант справедливости - вознаградит тех, кто был молодец, и накажет тех, кто не был молодец.
\end{enumerate}

\hfill

\textbf{Связь между теоретическим и практическим разумом}

\textbf{Противоречие} между гносеологией и свободой воли - мое моральное поведение \textbf{не более, чем конструкция моего разума} - свобода воли оказывается запертой в границах разума.

Свобода может быть только тем, что мы знаем, и ничем другим. Свобода оказывается \textbf{несвободной} - невозможной - этика оказывается невозможной.

Парадокс является \textbf{выражением абсолютной свободы} - может быть чем угодно - Кант сознательно его оставляет, спасает нашу возможность быть нравственными.

\paragraph{Критика способности суждения} Еще одна критика - эстетика и телеология - \textbf{понятие целесообразности}.

\paragraph{Просвещение} Просвещение - \textbf{публичное использование разума}

\end{flushleft}

\end{document}