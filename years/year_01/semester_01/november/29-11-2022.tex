\documentclass{article}
\usepackage[utf8]{inputenc}

\usepackage[T2A]{fontenc}
\usepackage[utf8]{inputenc}
\usepackage[polutonikogreek, english,russian]{babel}
\usepackage{csquotes}

\title{Философия}
\author{Лисид Лаконский}
\date{November 2022}

\begin{document}

\maketitle
\tableofcontents
\pagebreak

\section{Философия - 29.11.2022}

\subsection{Философия А. Шопенгауэра: философский пессимизм, воля и представление}

\subsubsection{Философский пессимизм}

\begin{flushleft}

Вся человеческая \textbf{жизнь — сплошные страдания и разочарования}. Желания никогда не удовлетворяются окончательно. Вскоре после достижения цели наступает равнодушие и скука. Между страданиями и скукой мечется человеческая жизнь.

Земное \textbf{счастье — иллюзия}. Счастье всегда находится в будущем (которое ненадежно) или в прошлом (которое уже невозможно), так что в настоящем человек никогда счастлив не бывает.

Мы не знаем трех высших благ жизни — здоровья, молодости и свободы. Пока они у нас есть, мы их не осознаем и не понимаем их ценности, а понимаем только тогда, когда утратим. Счастливые дни мы замечаем только тогда, когда они уступают место несчастным дням.

Все вокруг суетятся — одни в мечтах, другие в деятельности. Но последняя цель всего этого оказывается жалкой: поддержать на короткий промежуток времени жизнь.

Как бы ни была жалка жизнь, человек цепляется за нее из последних сил, воля к жизни сильнее разума, сам разум — продукт воли. Но эта \textbf{короткая отсрочка смерти имеет ничтожное значение — победа смерти несомненна}.

Поэтому человек должен выйти из-под власти воли, \textbf{подавить в себе всякие желания}: \textbf{понять, что страдания — неизбежная часть нашей жизни}, и если мы избавимся от одного, то неизбежно придет другое; если на время прекращаются страдания, то наступает скука, которая также является страданием. Если мы поймем это, нам удастся воспитать в себе равнодушие к страданиям. Ведь страдания, как и счастье, приходят не извне, а возникают изнутри человека. Подавить волю, перестать быть ее рабом, уменьшить тягостную заботу о собственном благополучии — таков единственно возможный путь мыслящего человека. Доступный, правда, немногим, кому дано понять, что \textbf{жизнь — это вечный обман и вечные разочарования}, что \textbf{в мире нет ничего достойного наших желаний, стремлений и борьбы, что все его блага ничтожны}. Тем самым мы преодолеваем господство воли.

\subsubsection{Учение о воле}

Шопенгауэр использовал слово \textbf{«воля»} как наиболее известное указание на концепцию, которую можно обозначить также словами \textbf{«вожделение», «стремление», «желание», «усилие», «призыв»}.

Вся природа, включая человека, - \textbf{это выражение ненасытной «воли к жизни»}.

Именно благодаря этому «желанию жить» человечество испытывает страдание. Желание большего — причина ещё больших страданий.

\subsubsection{Учение о представлении}

\textbf{Представление} - производимая в уме идея, образ или любой объект пережитого, в значении \textbf{находящегося снаружи по отношению к разуму}.

\pagebreak
\subsection{Философия Ф. Ницше: учение о воле}

\subsubsection{Учение о воле}

\textbf{Воля к власти} - это одна из разновидностей волевых импульсов человеческого поведения. Волю к власти Ницше считал \textbf{определяющим стимулом деятельности и главной способностью человека}. Основой жизни, по Ницше, является \textbf{воля к власти или тяга всего живого к самоутверждению, всемогуществу, стремлению расширить власть}. На протяжении всей жизни человек стремится достичь максимума чувства власти.

\textbf{Простые люди никчемны, слабы, половинчаты, мягкотелы}, не способны созидать и властвовать. Они – \textbf{рабы от природы} и могут лишь подчиняться. Надо любить не слабого и ближнего, а сильного

Простому человеку у Ницше противостоит сверхчеловек – существо высшего биотипа, не принадлежащее ни к одной расе и выращенное элитой общества. Сверхчеловек – это выражение полноты жизни, создатель новых ценностей. Он находится \textbf{вне всяких моральных норм, вне добра и зла, с особой жестокостью преодолевает тотальную ложь земного мира}.

Неоевропейскому рационализму Ницше противопоставил \textbf{чувства и инстинкты}, обеспечивающие, с его точки зрения, волю к власти. Он полагал, что разум по своей сути ничтожен, логика абсурдна, так как имеет дело с застывшими формами, противоречащими динамике жизни.

Главной целью познания Ницше считал \textbf{овладение миром}, а не установление истины, которая тождественна заблуждению. Заблуждение и ложь необходимы, поскольку утешают толпу и помогают аристократам духа самоутверждаться.

\pagebreak
\subsection{Философия А. Бергсона: жизненный порыв, деятельность}

\subsubsection{Учение о жизненном порыве}

\textbf{Жизненный порыв - метафора}, с помощью которой Бергсон сформулировал ряд важных для него \textbf{эволюционных идей}.

Жизнь \textbf{зарождается в одном центре} в силу начального импульса, а затем ее порыв продвигается по \textbf{множеству параллельных направлений}, претерпевая по пути целую серию качественных скачков, подобных взрывам. Из-за сопротивления, оказываемого материей, на большинстве линий порыв угасает и развитие прекращается, сменяясь круговоротом. 

\textbf{Среди направлений движения} порыва Бергсон \textbf{выделяет три основных – чувственность, инстинкт и интеллект}, которые приводят, соответственно, \textbf{к растениям, животным и человеку}.

\subsubsection{Учение о длительности}

Бергсон стремился устранить недостатки, которые он увидел в философии Герберта Спенсера, в связи, как он считал, с недостатком знаний Спенсера о механике, что привело Бергсона к выводу, что время ускользает от математики и естественных наук. Бергсон понял, что момент, когда человек пытался измерить время, ушёл: измеряется неподвижная, полная линия, тогда как время подвижное и неполное. В отдельных случаях, время может ускоряться или замедляться, тогда как для науки оно останется прежним. Таким образом, Бергсон решил исследовать внутренний мир человека, который является типом продолжительности, ни единство и не количественную множественность. Длительность невыразима и может быть отображена только косвенно, через образы, которые никогда не могут показать полную картину. Это может быть постигнуто только с помощью интуиции воображения.

Бергсон представляет три изображения длительности. Первое имеет две катушки: одна разворачивающаяся, чтобы представлять непрерывный поток старения, чувствуя своё приближение к концу продолжительности жизни, другая - сворачивающаяся, чтобы показать непрерывный рост памяти, которая, по мнению Бергсона, равняется сознанию. У человека без памяти могут возникнуть два одинаковых момента, но, говорит Бергсон, осознавая, что человек, таким образом, будет находиться в состояние смерти и возрождения, которую он отождествляет с потерей сознания. Изображение двух катушек, несмотря на то, что они из однородной и пропорциональной нити, в то время, как Бергсон считал, что никакие два момента не могут быть одинаковыми, поэтому продолжительность неоднородна.

Бергсон потом предоставил изображение спектра тысячи постепенно меняющихся оттенков с линией, которая проходит через них, находясь под влиянием и поддерживая каждый оттенок. Но даже этот образ является неточным и неполным, поскольку он представляет длительность в виде фиксированного и полного спектра всех оттенков, сопоставленных в пространстве, в то время как продолжительность неполная и постоянно растёт, её состояние это не начало и не конец, а что-то смешанное.

На самом деле, мы меняемся, не переставая… нет существенной разницы между переходом от одного состояния к другому и сохранения в том же состоянии. Если состояние, которое «остаётся неизменным» является более разнообразным, чем мы думаем, то с другой стороны, переход из одного состояния в другое напоминает больше, чем мы представляли — единое состояние продлевается: переход является непрерывным. Только потому, что мы закрываем наши глаза на непрерывное изменение каждого физического состояния, мы вынуждены, когда изменения стали настолько грозными, чтобы привлечь наше внимание, говорить, как если бы новое состояние было бы размещено рядом с предыдущими. Мы считаем, что это новое состояние, в свою очередь, остаётся неизменным и так до бесконечности.

Потому, что качественная кратность неоднородна и ещё проникающая себя, она не может быть адекватно представлена символом, да и для Бергсона, качественная кратность невыразима. Таким образом, чтобы понять длительность, нужно отойти от привычных способов мышления и поставить себя в течение длительности с помощью интуиции.

\pagebreak
\subsection{Рекомендуемая литература}

\begin{enumerate}
    \item Кротов А. А., Бугай Д. В. История философии. М.: Академический проспект, 2008
\end{enumerate}

\end{flushleft}

\end{document}