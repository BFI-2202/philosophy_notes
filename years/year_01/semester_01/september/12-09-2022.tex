\documentclass{article}
\usepackage[utf8]{inputenc}

\usepackage[T2A]{fontenc}
\usepackage[utf8]{inputenc}
\usepackage[russian]{babel}

\title{Философия}
\author{Лисид Лаконский}
\date{September 2022}

\begin{document}

\maketitle
\tableofcontents
\pagebreak

\section{Философия - 12.09.2022}

\subsection{Мировоззрение}

\begin{flushleft}

Типы мировоззрения: \textbf{мифологическое}, \textbf{религиозное}, \textbf{научное}, \textbf{философское}, \textbf{обыденное} (опирается на житейский опыт)

\subsection{Предмет философии}

Философия изучает взаимодействие между человеком и окружающим миром.

\textbf{Философия} - учение о наиболее общих законах природы, общества и мышления человека.

Философия - школа мышления.

\subsection{Функции философии}

Выделяют следующие функции философии:

\begin{itemize}
    \item Мировоззренческая - система аргументированных взглядов на мир и его явления, свойства и взаимоотношения.
    \item Познавательная
    \item Методологическая - выработка принципов отношения человека к миру.
    \item Аксиологическая - оценка всех явлений и того, какое значение они имеют в жизнедеятельности людей
    \item Мотивационно-побудительная
\end{itemize}

\subsection{Еще одно определение философии}

\textbf{Философия} - особая форма познания и система знаний об общих характеристиках, понятиях и принципах бытия, а также бытия человека, об отношениях человека и окружающего его мира.

\end{flushleft}

\end{document}