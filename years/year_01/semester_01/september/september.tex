\documentclass{article}
\usepackage[utf8]{inputenc}

\usepackage[T2A]{fontenc}
\usepackage[utf8]{inputenc}
\usepackage[russian]{babel}

\title{Философия}
\author{Лисид Лаконский}
\date{September 2022}

\begin{document}

\maketitle
\tableofcontents
\pagebreak

\section{Философия — 29.09.2022}

\subsection{Греческая философия. Досократическая греческая философия}

\begin{flushleft}
Греция. Малая Азия - Иония. Слева от Греции - Италия. Досократическая греческая философия - философия древнегреческих колоний.

Философские школы досократического периода по географическому положению:

\begin{enumerate}
    \item Ионийские
    \item Италийские
\end{enumerate}

До периода досократического - Гомер, Гесиод. Архаические времена

\end{flushleft}

\subsubsection{Милетская школа - 7 век до н.э}

\begin{flushleft}

\textbf{Представители}: Фалес Милетский, Анаксимандр, Анаксимен, Гераклит Эфесский\par

\textbf{Общая характеристика периода}: объединена одной темой: натурофилософией. Философия природы. Главные темы, поднимаемые в досократической мысли: космогония — происхождение мира, космология - наука об устройстве мира. Антропоморфизм.

\hfill

Философия - не миф в чистом виде, попытка рационализации мифа. Стиль философствования сильно отличался от послесократического. Главный метод - созерцание. Созерцательная философия. Эмоционально вовлеченное наблюдение. Тождественность человека и космоса.

\hfill

\textbf{Центральное понятие} во всей милетской философии - понятие «архэ», первый элемент, из чего все произошло. Фалес в качестве «архэ» принял воду: все произошло из воды. Вода дает мне жизнь - вода дает жизнь космосу вообще - антропоморфизм. Древние ведали, что человек в основном состоит из воды. Фалес пытается объяснить, почему именно вода - чисто философская мысль.

\hfill

Разные вещи в мире обладают различной формой. Если предполагаем, что все из одного элемента - первоэлемент должен обладать свойством аморфности - вода идеально подходит. Объяснение многообразия форм в мире.

\hfill

Космологию напрямую заимствует у Гомера - Иллиада - диск, лежащий на воде, подобно дощечке - следствие созерцательного мышления.

\hfill

\textbf{Фалес} объяснял неподвижность почв тем, что земля слишком тяжелая, чтобы ветер мог ее сдвинуть. Волны - землетрясения.

\hfill

\textbf{Анаксимандр} - ученик Фалеса, занят той же проблемой, происхождением мира. В отличие от учителя в качестве «архэ» выбрал «апейрон» (беспредельное божественное) - мы не знаем, что это такое. Анаксимандр описывает «апейрон» как нечто среднее между воздухом и огнем.

\hfill

Представители милетской школы не были материалистами. Все произошло из «воды» - не в прямом смысле, божественная стихия.

\hfill

Своя космология. Земля как столб высотой четырех ее диаметров. Подвешена в пространстве, окружена апейроном.

\hfill

Отход от мифологического мышления: «апейрон» беспределен - следовательно, и миров тоже бесконечное множество. Все из «апейрона» произошло - все в «апейрон» вернется — цикличность мира.

\hfill

У \textbf{Гесиода} история линейна и регрессивна. Легенда - первый начал писать произведения в прозе.

\hfill

\textbf{Анаксимен} - ученик Анаксимандра, для него первым элементом был воздух. Анаксимен вернулся обратно к Фалесу и выбрал в качестве первоосновы достаточно известную стихию, причем на тех же основаниях: воздух аморфен.

\hfill

Космогония Анаксимена: земля похожа на усеченную трапецию, с одной стороны поднятую.

\hfill

\textbf{Гераклит Эфесский} - досократик. Все в мире произошло из борьбы: «война есть отец всего».

\hfill

Стихийный диалектик. Все сущее - результат борьбы противоположностей. Результат борьбы - единство. Все, что нас окружает, произошло из борьбы. Первоэлемент, из которого все произошло - огонь (божественный). Мировой пожар - огонь погас, затух - разложился на остальные элементы. «Путь вниз» - из огня появилось все. «Путь вверх» - мировой пожар - потух - появились всякие вещи - потом сгорят снова в мировом огне - так далее.

\hfill

Циклическая космология. Мировой закон - «тайная гармония» - логос. Циклы возгорания и угсания имеют строго физиологическую природу - дыхание природы.

\end{flushleft}

\subsubsection{Элейская школа}

\begin{flushleft}

Основатель элейской школы - \textbf{Парменид}. Италийская философия. Процесс смены мышления.

\hfill

Парменид все еще занимался космосом, но созерцание отошло на второй план. Главный предмет интереса - бытие.

\hfill


Главный и основной тезис: «бытие есть, а небытия нет», «бытие это то, что есть, и то, что может быть», «небытие - этого то, чего нет, и чего может не быть»

\hfill

Бытие не может не быть, так как оно есть. Небытие - наоборот, может не быть. Бытие - нечто существующее, небытие - это нечто несуществующее.

Для Парменида бытие - это шар. Шар, потому что шар - совершенная форма. У бытия должна быть совершенная форма. Бытие беспредельно: шарик бесконечного размера, т.к. бытия есть, а небытия нет. Если бы шарик имел конкретные размеры - за его границами было бы небытия, а небытия нет.

\hfill

Бытие \textbf{вечно}: всегда было и всегда будет, потому что если бы оно когда-то появилось - это означало бы, что когда-то было небытие. Если бы исчезло - после него было бы небытие, что невозможно. Бытие \textbf{неподвижно}, так как оно бесконечно размерное; если бы бытие сдвинулось, это бы значило, что бытию было куда двигаться - необходима пустота - небытие - невозможно.

\hfill

Из того, что бытие неподвижно, Парменид говорит о том, что \textbf{движения нет}. То, что что-то двигается - иллюзия восприятия, «путь мнения». Нам просто кажется, что в мире есть движение, так как чувства обманчивы.

\hfill

Противопоставление «пути мнения» - путь истины - путь логического доказательства. Только на логическое доказательство можно опираться. Все выводы Парменида - логические умозаключения, поэтому они истинны.

\hfill

Парменид \textbf{отходит от созерцательной философии}. На чувственность опираться невозможно - строго логический метод. Логика выше чувственности. Рациональное выше эмпирического.

\hfill

Ученик - \textbf{Зенон Элейский}. Апории Зенона, от греческого «затрудения».

\hfill

\textbf{Демокрит Абдерский} - территория современной Болгарии

Демокрит - представитель классической эллинской философии. Но его философия по содержанию досократическая. Последний представитель досократической теории.

\hfill

\textbf{Все состоит из атомов и пустоты}. Есть только атомы и пустота. Субстанциальная концепция пространства - пустота как субстанция.

\hfill

Атомы, про которые рассказывает Демокрит - корпускулы, частицы. Обладают разной формой, разным цветом и разным размером. Имеют разные поверхности на ощупь.

\hfill

Во вселенной существует бесконечное число атомов - в ней существует бесконечное число миров. Теория множественности миров. Так как из атомов и пустоты состоит вообще все, то боги тоже состоят из атомов - бесконечное множество богов.

\hfill

Атомы сцепляются друг с другом, образуя вещи. Зазоры между атомами - пустота. Материя не делима бесконечно. 

Сохраняются мифологические мотивы. Космос бесконечный, бесконечное количество атомов. Вихрь из атомов бесконечного размера. Атомы сцепляются, появляются миры, появляются боги...

\hfill

\textbf{Гноселогия} - проблемы познания - тоже занимался этим. Все состоит из атомов - следовательно, наши органы чувств тоже состоят из атомов. От предмета во все стороны истекают атомы - эти атомы попадают в глаза, уши - мы видим и слышим вещи. Мы можем воспринимать только то, атомы которых родственны и подобны атомам нашего собственного тела. Глаз, сделанный из атомов ABC, может видеть только атомы ABC - вывод о том, что в мире есть вещи воспринимаемые, а есть вещи невоспринимаемые. Мы видим только часть мира.

\hfill

Предпосылок для атомиза в Древней Греции не было. Демокрит - Абдеры - берег Эгейского мира. Из богатой семьи и унаследовал кучу денег - отправлялся в путешестии на Восток - доехал до Персии и Египта. Возможно, доехал до Индии. Товарообмен, обмен знаниями. Возможно, в своих путешествиях встретил кого-то из Индии (учился в Индии) - в Индии атомизм появился еще в шестом веке до нашей эры.

\hfill

Есть легенда, рассказывающая о том, что когда Демокрит возвратился домой из путешествия, его отдали под суд за растрату семейных денег. Демокрит, выступая на суде, защищал себя сам: «я не растратил отеческие деньги впустую, так как я принес с востока знание». Судьи согласились, что Демокрит верно потратил деньги и дали ему еще денег в качестве награждения.

\end{flushleft}

\end{document}