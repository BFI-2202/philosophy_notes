\documentclass{article}
\usepackage[utf8]{inputenc}

\usepackage[T2A]{fontenc}
\usepackage[utf8]{inputenc}
\usepackage[russian]{babel}

\title{Философия}
\author{Лисид Лаконский}
\date{September 2022}

\begin{document}

\maketitle
\tableofcontents
\pagebreak

\section{Философия - неизвестная дата}

\begin{flushleft}

\subsection{Разделы философии}

\begin{itemize}
    \item Онтология - учение о бытии; о том, как устроен мир, его тенденциях. Все тезисы - допущения
    \item Гносеология - изучение природы знания
    \item Аксиология - изучение ценности
    \item Теология - изучение бога (не богословие - в богословии догматы)
    \item Этика - вопросы морали: что есть добро, что есть зло. Философия поступка. Человеческий поступок
    \item Эстетика - проблемы прекрасного и безобразного (совр. философия искусства)
\end{itemize}

\subsection{Мировоззрение}

\textbf{Мировоззрение} - система взглядов, ценностей, убеждений, через которую мы смотрим на мир.

Типы мировоззрения:

\begin{itemize}
    \item Мифологическое
    \item Религиозное
    \item Научное
\end{itemize}

\textbf{Мифологическое мировоззрение} основано на мифах. Характерен антропоморфизм.

\textbf{Космогония} - миф о происхождении мира.

\hfill

\textbf{Религиозное мировоззрение} объясняет мир исходя из существования бога. (надмирового существа)

\hfill

\textbf{Научное мировоззрение} объясняет мир исходя из логических и эмпирических доказательств.

\end{flushleft}

\end{document}