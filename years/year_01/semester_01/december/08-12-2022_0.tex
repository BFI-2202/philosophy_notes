\documentclass{article}
\usepackage[utf8]{inputenc}

\usepackage[T2A]{fontenc}
\usepackage[utf8]{inputenc}
\usepackage[polutonikogreek, english,russian]{babel}
\usepackage{csquotes}

\usepackage{hyperref}
\hypersetup{
    colorlinks, citecolor=black, filecolor=black, linkcolor=black, urlcolor=black
}

\newtheorem{definition}{Определение}

\title{Философия}
\author{Лисид Лаконский}
\date{December 2022}

\begin{document}

\maketitle
\tableofcontents
\pagebreak

\section{Философия - 08.12.2022}

\subsection{Онтология}

\begin{flushleft}

\begin{definition}
Онтология (новолат. ontologia от др.-греч. \textgreek{ὄν}, род. п. \textgreek{ὄντος} — сущее, то, что существует + \textgreek{λόγος} — учение, наука) — учение о бытии.
\end{definition}

\begin{definition}
Онтология — совокупность некоторых общих допущений (эмпирические не верифицируемых) о характере мира.
\end{definition}

Пример онтологических допущений: мир бесконечен в пространстве (мир конечен в пространстве), мир имеет конец во времени (не имеет конца во времени)

Пример современной онтологии: мир имел начало во времени, бесконечен в пространстве, имеет конец во времени, упорядочен (существует причинность и так далее) — физика.

\subsection{Категории бытия}

\subsubsection{Бытие}

Как определять бытие не надо:

\begin{enumerate}
    \item Бытие — это всё
    \item Бытие — это способность к существованию
\end{enumerate}

Более вменяемое определение: \textbf{бытие} — понятие самой высокой степени абстракции; понятие всех понятий; понятие, которое включает в себя все остальные понятия

\subsubsection{Существование}

Почему вообще в мире что-то есть? Наиболее удачное определение:

С точки зрения логики, \textbf{существовать} — значит быть значением квантифицированной переменной; то есть, значением переменной, к которой был приписан квантор существования.

Две \textbf{формы бытия}:

\begin{enumerate}
    \item Материальное — можно эмпирически верифицировать.
    \item Идеальное — обладают объекты духовной культуры.
\end{enumerate}

Иначе выделяют две формы бытия: \textbf{социальное} и \textbf{человеческое} (не рекомендуется говорить об этом на экзамене, лучше скажите про материальное и идеальное).

Человеческое бытие — \textbf{экзистенция}, иногда выделяют как отдельную форму бытия.

\pagebreak
\subsection{Онтология диалектического материализма}

Исток диалектического материализма (далее — диамат) — \textbf{середина девятнадцатого века}

\begin{definition}
    Материя — объективная действительность (то, что существует независимо от нашего сознания); центральная категория диалектического материализма
\end{definition}

Согласно диамату, \textbf{сознание — тоже материя}, так как имеет материальный субстрат — мозг. Понимание материи в диамате \textbf{отличается} от понимания материи в современной философии и современной науке (до какой степени материальна энергия [поле] в физике)

\subsubsection{Аттрибуты материи}

\begin{enumerate}
    \item \textbf{Движение} — \textbf{всякое изменение} (становление, как это называли древние греки); возможно благодаря тому, что в мире существует множество объектов; имманентный аттрибут - присущий самой материи. Пять форм движения материи (в иерархическом порядке, от низшего к высшему):
    \begin{enumerate}
        \item \textbf{механическая} — перемещение объекта в пространстве
        \item \textbf{физическая} — уровень физического взаимодействия; уровень влияния всяких физических взаимодействий: инерции, силы трения, всяких других сил и так далее
        \item \textbf{химическая} — конкретное взаимодействие веществ
        \item \textbf{биологическая} — движение живой материи (материи в живых существах); все процессы в живых существах
        \item \textbf{социальная}; общество — чисто материальный субстрат, так как общество невозможно без людей — материальных сущностей; общественные процессы протекают независимо от нашего сознания — следовательно, объективны — по определению, материя; то есть, по диамату, власть материальна
    \end{enumerate}
    \item \textbf{Пространство} — способ существования материи;
    \item \textbf{Время} — способ существования материи; как это называет Аристотель, \textbf{число изменения} (число движения)
\end{enumerate}

Пространство и время можно определять \textbf{реляционно} (в зависимости от других понятий) и \textbf{субстанциально}; реляционные определения преобладают, но субстанциальные тоже существуют: Демокрит (мир состоит из атомов и пустоты), Платон, Ньютон (пространство однородно по всей вселенной, «коробочка» бесконечного размера, убрать содержание — коробочка останется — абсолютное пространство; время для любого объекта во вселенной течет равномерно). Можно также отдельно выделить \textbf{психологические} концепции.

\subsubsection{Уровни организации материи}

\begin{definition}
    Микромир — все, что мы не можем разглядеть глазом
\end{definition}

\begin{definition}
    Макромир — человеко-размерный мир; размеры которого представимы относительно человека; объекты на поверхности Земли
\end{definition}

\begin{definition}
    Мегамир — все, размеры чего несопоставимы с размером человека
\end{definition}

\subsubsection{Структура организации материи}

\paragraph{Неживая природа:} субатомные частицы; атомы; молекулы; вещество и макрообъекты; планеты; скопление планет; галактики...

\paragraph{Живая природа:} белки; белковые цепи; одноклеточные организмы; многоклеточные организмы; популяция; биоценоз; биосфера — совокупность всей жизни на Земле

\paragraph{Общество:} индивид; семья; коллектив; общественный класс; нация; наднациональные образования (международные организации); человечество вооббще

\end{flushleft}

\end{document}