\documentclass{article}
\usepackage[utf8]{inputenc}

\usepackage[T2A]{fontenc}
\usepackage[utf8]{inputenc}
\usepackage[polutonikogreek, english,russian]{babel}
\usepackage{csquotes}

\usepackage{hyperref}
\hypersetup{
    colorlinks, citecolor=black, filecolor=black, linkcolor=black, urlcolor=black
}

\usepackage{graphicx}
\graphicspath{ {./images/} }

\newtheorem{definition}{Определение}

\title{Философия}
\author{Лисид Лаконский}
\date{December 2022}

\begin{document}

\maketitle
\tableofcontents
\pagebreak

\section{Философия - 22.12.2022}

\subsection{Возникновение сознания и его социальная природа. Сознание и мозг}

\pagebreak
\subsection{Сознание и язык. Виды и функции языка}

\begin{flushleft}

\textbf{Сущность языка} выявляется в его двуединой функции: служить средством общения и орудием мышления. \textbf{Речь} — это деятельность, сам процесс общения, обмена мыслями, чувствами, пожеланиями, целеполаганиями и т.п., который осуществляется с помощью языка, т.е. определенной системы средств общения.

\textbf{Язык} — это система содержательных, значимых форм. Язык выполняет роль механизма \textbf{социальной наследственности}.

Сознание и язык образуют \textbf{единство}: в своем существовании они предполагают друг друга, как внутреннее, логически оформленное идеально содержание предполагает свою внешнюю материальную форму. Язык есть \textbf{непосредственная деятельность мысли, сознания}. Он учавствует в процессе мыслительной деятельности как ее чувственная основа или орудие. Сознание не только выявляется, но и \textbf{формируется} с помощью языка. Мысли \textbf{строятся в соответствии с языком} и должны ему соответствовать. Справедливо и обратное: \textbf{мы организуем нашу речь в соответствии с логикой нашей мысли}.

Посредством языка происходит \textbf{переход от восприятий и представлений} к понятиям, протекает процесс оперирования понятиями. В речи человек фиксирует свои мысли, чувства и благодаря этому имеет \textbf{возможность подвергать их анализу} как вне его лежащий объект. Язык и сознание \textbf{едины}. В этом единстве определяющей стороной является сознание: будучи отражением действительности, оно «лепит» формы и диктует законы своего языкового бытия. Но единство — не тождество: сознание \textbf{отражает} действительность, а язык \textbf{обозначает} ее.

Язык и сознание образуют \textbf{противоречивое единство}. Язык влияет на сознание: его исторически сложившиеся нормы, специфичные у каждого народа, в одном и том же объекте оттеняют различные признаки. Однако зависимость мышления от языка \textbf{не является абсолютной}: мышление детерминируется главным образом своими связями с действительностью, язык же может лишь \textbf{частично модифицировать форму и стиль мышления}.

\end{flushleft}

\pagebreak
\subsection{Сознание и самосознание}

\subsubsection{Теория о первичности сознания себя}

\begin{flushleft}

По \textbf{В.М. Бехтереву} (Владимир Михайлович Бехтерев), простейшей формой сознания следует признавать то состояние, когда еще не выработано ни одного более или менее ясного представления и суещствует лишь \textbf{неясное безотносительное чувствование собственного существования}.

Более сложным является сознание, в котором уже присутствуют те или иные представления. Но и в этом случае \textbf{наиболее элементарной формой} сознания следует признавать ту, при которой в сознании присутствует \textbf{главным образом одна группа представлений} о «Я» как субъекте в отличие от «не—Я» или объекта и из которой вырабатывается так называемое самосознание.

Лишь после этого возникают те формы сознания, которые основываются на пространственных и временных представлениях, и, наконец, как синтез первичного предметного созанния — формы, которые составляют «интимное ядро личности» и при которых человек \textbf{может анализировать} происходящие в нем самом психические процессы.

Для этой точки зрения характерно утверждение, что познание ребенком самого себя на первых этапах развития протекает \textbf{в отрыве от познания им окружающего мира} и даже в \textbf{отрыве от собственной деятельности}. Оно поддерживается исключительно за счет тех органических ощущений и чувствований, которые \textbf{даны ребенку от рождения}. Эти ощущения и чувства, таким образом перестают быть предпосылками в развитии самосознания и становятся \textbf{его источником и движущей силой}.

\end{flushleft}

\subsubsection{Теория об единстве развития сознания и самосознания}

\pagebreak
\subsection{Список литературы}

\subsubsection{Рекомендуемая литература}

\begin{enumerate}
    \item Алексеев П.В., Панин А.В. Философия. М.: Издательство Проспект, 2005
    \item Спиркин А.Г. Философия: учебник. М. Гардарики, 2006
\end{enumerate}

\end{document}