\documentclass{article}
\usepackage[utf8]{inputenc}

\usepackage[T2A]{fontenc}
\usepackage[utf8]{inputenc}
\usepackage[polutonikogreek, english,russian]{babel}
\usepackage{csquotes}

\usepackage{hyperref}
\hypersetup{
    colorlinks, citecolor=black, filecolor=black, linkcolor=black, urlcolor=black
}

\usepackage{graphicx}
\graphicspath{ {./images/} }

\newtheorem{definition}{Определение}

\title{Философия}
\author{Лисид Лаконский}
\date{December 2022}

\setcounter{tocdepth}{4}
\setcounter{secnumdepth}{4}

\begin{document}

\maketitle
\tableofcontents
\pagebreak

\section{Философия - 22.12.2022}

\subsection{Философия сознания}

\subsubsection{Историческая справка}

\begin{flushleft}

Уже в Древней Греции поднимались проблемы души (сознания) — \textbf{субстанциальное понимание} — как особого рода субстанции, строго отделенные от тела.

Сознание и тело онтологически отделены — \textbf{два разного рода субстанции}.

С появлением христианства \textbf{понимание о субстанциальное душе остается} — наследие древнегреческой мысли. Прожило до восьмидесятых годов двадцатого века.

Семнадцатый век — Декарт — базовые установки в теме сознания (мышления): cogito ergo sum, «центр мышления» — \textbf{мышление локализовано}, «шишковидная железа» — \textbf{хранилище души}.

Иммануил Кант — \textbf{трансцендентальный субъект} — центр познания.

\subsubsection{Современное положение}

Нейробиология положила конец картезианскому мифу о локализованном сознании.

Дэниел Деннет (основатель \textbf{функционализма}) — развенчал представление о «картезианский театр» — неправда, зрителя нет. У мозга есть \textbf{функции}; следует изучать сознание как набор функций.

\hfill

Аналитическая философия — \textbf{атлантическая школа}, тесная связь с нейронауками — философия сознания. Также есть континентальная школа. Некоторые буддистсткие школы заявляли об \textbf{отсутствии сознания}.

Дэвид Чалмерс — \textbf{легкие} (может быть решена эмпирически) и \textbf{трудные} (нерешаемая эмпирическими средствами) проблемы сознания: «каков характер связи между состояниями мозга и состояниями сознания?»

Есть несколько подходов к решению данной проблемы:

\begin{enumerate}
    \item Ее отмена — например, буддисты, исследователи-материалисты
    \item Непризнание ее сложной проблемой
    \item Попытки разложения ее на простые проблемы
\end{enumerate}

\paragraph{Вопрос о необходимости сознания} Зачем необходимо сознание? Миллионы биологических видов прекрасно живут без него. Концептуально это неясно.

\paragraph{Космологический подход} \textbf{Сильный антропологический принцип} — сознание \textbf{появилось по необходимости}, материя обладает потенцией к сознанию — потенция рано или поздно реализовывается.

\textbf{Слабый антропологический принцип} — появление мышления строго случайно.

\paragraph{Великая иллюзия сознания} Мы своего рода лишь \textbf{наблюдатели} собственной жизни — не непосредственные участники — \textbf{феноменальное сознание} (сознание ощущений) — очень малая часть того, что воспринимает мозг.

Мы не контролируем большую часть своей жизни — поддержание равновесия, пищеварение и так далее.

\paragraph{Проблема исследования сознания} Исследование сознания \textbf{ин-витро невозможно}, лишь \textbf{ин-виво} — методом интроспекции и самоотчета, что \textbf{необъективно}.

Мы можем делать выводы о сознании лишь исходя из степени правдоподобия.

\end{flushleft}

\end{document}