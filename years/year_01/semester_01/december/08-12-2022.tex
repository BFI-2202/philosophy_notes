\documentclass{article}
\usepackage[utf8]{inputenc}

\usepackage[T2A]{fontenc}
\usepackage[utf8]{inputenc}
\usepackage[polutonikogreek, english,russian]{babel}
\usepackage{csquotes}

\usepackage{hyperref}
\hypersetup{
    colorlinks, citecolor=black, filecolor=black, linkcolor=black, urlcolor=black
}

\newtheorem{definition}{Определение}

\title{Философия}
\author{Лисид Лаконский}
\date{December 2022}

\begin{document}

\maketitle
\tableofcontents
\pagebreak

\section{Философия - 08.12.2022}

\subsection{Экзистенциальная философия Ж.—П. Сартра}

\begin{flushleft}

\textbf{Основой} философии Сартра выступает \textbf{проблема понимания человеческого бытия как сознательной, свободной деятельности}. Ограничение интереса философа вопросами духовной жизни людей в их повседневном быту объясняется тем, что он рассматривает сферу трудовой экономической деятельности как область, в которой человек ведет \textbf{неподлинное существование} (не принадлежит себе, подчиняется навязанным ему нормам). Реакцией на подобное состояние у героев произведений Сартра является чаще всего затворничество или бегство от неприемлемой действительности.

\hfill

В главном уже собственно философском труде Сартра «Бытие и ничто» делается попытка выяснить суть бытия, обусловливающего неподлинность существования.

\hfill

Согласно представлениям Сартра, субъективность отдельного сознания становится бытием для других, когда \textbf{существование личности попадает в область восприятия другого сознания}. При этом отношение к другому представляет собой \textbf{борьбу за признание свободы личности со стороны другого человека}.

\hfill

Человеческое существование, полагал Сартр, есть \textbf{последовательная цепь самоотрицаний, в которых находит реализацию свобода}.

\begin{definition}
    Человеческое существование — последовательная цепь самоотрицаний, в которых находит реализацию свобода
\end{definition}

Человеку \textbf{изначально присуща свобода}, не терпящая ни причин, ни оснований, и \textbf{не определяющаяся ни прошлым, ни настоящим}. Свобода означает разрыв с ними и отрицание их. Быть свободным — значит \textbf{иметь возможность изменяться и обладать способностью действовать в мире}.

\begin{definition}
    Свобода означает разрыв с прошлым и настоящим, отрицание их; быть свободным — значит иметь возможность изменяться и обладать способностью действовать в мире
\end{definition}

Для Сартра \textbf{человек обладает свободой независимо от реальных возможностей} реализации своих желаний. По мнению философа, объективные обстоятельства не могут лишить человека свободы. Она может сохраняться в любых условиях и \textbf{представляет собой возможность выбора отношений к явлениям окружающей действительности}.

\begin{definition}
    Свобода представляет собой возможность выбора отношений к явлениям окружающей действительности
\end{definition}

Так, например, узник может смириться со своим положением, а может бунтовать против насилия и умереть непокоренным. Такое понимание свободы вытекало из отрицания каких-то раз и навсегда данных оснований свободы. \textbf{Свобода ставится в зависимость от окружающих человека обстоятельств и от их понимания человеком}.

\hfill

Согласно Сартру, \textbf{перед лицом мира человек испытывает одиночество}, которое становится условием не только страдания, но и \textbf{средством, указывающим ему место в мире}, \textbf{наделяющим его позицией, правами и обязанностями}.

\begin{definition}
    Одиночество — условие страдания; средство, указывающее человеку место в мире, наделяющее его позицией, правами и обязанностями
\end{definition}

Человек, будучи заброшенным в мир, \textbf{испытывает также тоску и тревогу и посредством их сознает свою свободу}. Человек \textbf{оказывается свободным в любых обстоятельствах}. Свобода превращается в роковое бремя, от которого невозможно избавиться. Свобода желать у Сартра — ее высшее проявление. Сартровское понимание свободы \textbf{предоставляет равноправные возможности для самых различных линий поведения}. Абсолютизация философом принадлежности свободы человеку проявляется в оправдании любых способов ее реализации в поведении, выражающихся в стойкости, самопожертвовании, великодушии, а также в аполитичности, предательстве, насилии и тому подобное.

\hfill

Сартр \textbf{считал экзистенциализм выражением гуманизма}, так как именно он, по его мнению, выступает в качестве той философии, которая \textbf{напоминает «человеку, что нет другого законодателя, кроме него самого}, и что решать свою судьбу он будет в полном одиночестве». Однако \textbf{экзистенциализм — «это не попытка отбить у человека охоту к действиям}, ибо он говорит человеку, что надежда лишь в его действиях и что \textbf{единственное, что позволяет человеку жить, — это действие}».

\hfill

Сартровская концепция свободы предопределяет характер его этики. В \textbf{фундамент нравственности} он положил \textbf{свободное волеизъявление личности}. \textbf{Личная свобода} человека рассматривается им как \textbf{единственная основа ценности и неценности поступков}. В качестве критерия моральности представлений личности Сартр выделяет их «аутентичность», т. е. \textbf{соответствие их подлинным представлениям}, свойственным моральному сознанию человека. Согласно Сартру, «... хотя содержание морали и меняется, \textbf{определенная форма этой морали универсальна}».

\hfill

Наделяя людей свободой, философ возлагает на них и \textbf{безусловную ответственность}. Действие последней находит свое выражение в \textbf{критическом отношении к миру и людям}, в \textbf{ощущении тревоги} в осуждении несправедливости и насилия, в \textbf{желании освободиться} от пагубного влияния окружения даже путем обречения себя на одиночество и скитания. Философ писал, что он на стороне тех, кто хочет изменить и условия жизни, и самого себя.

\hfill

Как философу, Сартру был присущ поиск теории, которая бы дала возможность прояснить \textbf{обстоятельства существования свободной деятельности людей, способной изменить ситуации их жизни и привести к свободе}.

\hfill

Жан-Поль Сартр \textbf{видел в культурной деятельности средство улучшения жизни}. И хотя «культура ничего и никого не спасает, да и не оправдывает, но \textbf{она — создание человека}: он себя проецирует в нее, узнает в ней себя; только в этом критическом зеркале видит он свой облик». 

Философ желал лишь изобразить мир человеческих отношений во всей его неприглядности, чтобы \textbf{помочь другим отразить его правильнее и при этом становиться лучше}.

Жан-Поль Сартр творил в надежде, что в урочный час, когда зловещие сумерки кризиса, опустившегося на Европу, подчиняясь непреложным законам бытия, начнут редеть и забрезжит свет нового светлого дня человечества, люди, учитывая опыт прошлого, быстрее поймут, какими им следует быть и что им необходимо делать.

\end{flushleft}

\pagebreak
\subsection{Экзистенциальная философия А. Камю}

\begin{flushleft}

Камю не сомневался в реальности мира, отдавал он себе отчет и в важности движения в нем. Мир, по его мнению, не устроен разумно. Он \textbf{враждебен человеку} и \textbf{все, что мы о нем знаем, малодостоверно}. Мир постоянно \textbf{ускользает от нас}. В своем представлении о бытии философ исходил из того, что «\textbf{бытие может выявить себя только в становлении}, становление же ничто без бытия». Бытие \textbf{отражается в сознании}, но «до тех пор, пока разум безмолвствует в неподвижном мире своих надежд, все взаимно перекликается и упорядочивается в столь желанном ему единстве. Но при первом же движении весь этот мир трещит и разрушается: познанию предлагает себя бесконечное множество мерцающих осколков». Познание Камю рассматривает как \textbf{источник преобразования мира}, но он предостерегает от неразумного использования знаний.

\hfill

Философ соглашался с тем, что \textbf{наука углубляет наши знания о мире и человеке}, но он указывал на то, что эти знания всё еще остаются несовершенными. По его мнению, наука до сих пор не дает ответа на самый настоятельный вопрос — вопрос о цели существования и смысле всего сущего. Люди заброшены в этот мир, в эту историю. Они смертны, и \textbf{жизнь предстает перед ними как абсурд в абсурдном мире}. Что же делать человеку в таком мире? Камю предлагает \textbf{сконцентрироваться} и с максимальной ясностью ума \textbf{осознать выпавший удел и мужественно нести бремя жизни, не смиряясь с трудностями и бунтуя против них}. При этом вопрос о смысле жизни приобретает особое (неотложнейшее) значение. С самого начала человек \textbf{должен «решить, стоит или не стоит жизнь того, чтобы ее прожить».} Ответить на этот «фундаментальный вопрос философии» — значит решить серьезную философскую проблему. По мнению Камю, все остальное второстепенно. \textbf{Стремление жить, полагает философ, диктуется привязанностью человека к миру}, в ней «есть нечто более сильное, чем все беды мира». Эта привязанность дает человеку \textbf{возможность преодолеть разлад между ним и жизнью}. Ощущение этого разлада порождает чувство абсурдности мира. Человек, будучи разумным, стремится упорядочивать, «преобразовывать мир в соответствии со своими представлениями о добре и зле. \textbf{Абсурд соединяет человека с миром}».

\hfill

Альбер Камю считал, что \textbf{жить означает исследовать абсурд, бунтовать против него}. «Я извлекаю из абсурда, — писал философ, — три следствия — \textbf{мой бунт, мою свободу и мою страсть}. Посредством одной только работы ума я обращаю в правило жизни то, что \textbf{было приглашением к смерти}, — и \textbf{отвергаю самоубийство}».

\hfill

По мнению А. Камю, человек имеет выбор: либо \textbf{жить в своем времени, приспосабливаясь к нему}, либо \textbf{пытаться возвыситься над ним}, но можно и вступить с ним в сделку: «\textbf{жить в своем веке и веровать в вечное}». Последнее не импонирует мыслителю. Он считает, что от абсурда можно заслониться погружением в вечное, спастись бегством в иллюзии повседневности или следованием какой-то идее. Иными словами, \textbf{снизить давление абсурда можно с помощью мышления}.

\hfill

Людей, пытающихся возвыситься над абсурдом, Камю называет \textbf{завоевателями}. Согласно Камю, \textbf{завоеватель богоподобен}, «он \textbf{знает свое рабство и не скрывает этого}», путь его к свободе освещает \textbf{знание}. Завоеватель — это идеал человека для Камю, но быть таковым, по его мнению, — это удел немногих.

\hfill

В абсурдном мире абсурдно и творчество. Согласно Камю, «\textbf{творчество — наиболее эффективная школа терпения и ясности}. Оно является и потрясающим свидетельством единственного достоинства человека: упорного бунта против своего удела, настойчивости в бесплодных усилиях. Творчество требует каждодневных усилий, владения самим собой, точной оценки границ истины, требует меры и силы. Творчество есть род аскезы. И все это «ни для чего»... Но может быть \textbf{важно не само великое произведение искусства, а то испытание, которого оно требует от человека}».

\hfill

Размышляя о своем времени как о времени торжества абсурда, Камю пишет: «Мы живем в эпоху мастерски выполненных преступных замыслов». Предшествующая эпоха, по его мнению, отличается от нынешней тем, что «\textbf{раньше злодеяние было одиноким}, словно крик, а \textbf{теперь оно столь же универсально, как наука}. Еще вчера преследуемое по суду, сегодня преступление стало законом». Философ отмечает: «В новые времена, когда злой умысел рядится в одеяния невинности, по страшному извращению, характерному для нашей эпохи, именно невинность вынуждена оправдываться». При этом \textbf{граница между ложным и истинным размыта, и правила диктует сила}. В этих условиях люди делятся «не на праведников и грешников, а на господ и рабов». Камю полагал, что \textbf{в нашем мире господствует дух нигилизма}. Осознание несовершенства мира порождает \textbf{бунт}, \textbf{цель которого — преображение жизни}. Время господства нигилизма формирует бунтующего человека.

\hfill

Согласно Камю, \textbf{бунт — это не противоестественное состояние, а вполне закономерное}. По его мнению, «\textbf{для того чтобы жить, человек должен бунтовать}», но делать это надо, \textbf{не отвлекаясь от первоначально выдвинутых благородных целей}. Мыслитель подчеркивает, что в опыте абсурда страдание имеет индивидуальный характер, в бунтарском же порыве оно становится коллективным. Причем «зло, испытанное одним человеком, становится чумой, заразившей всех».

\hfill

В несовершенном мире бунт выступает \textbf{средством предотвращения упадка общества и его окостенения и увядания}. «Я бунтую, следовательно, мы существуем», — пишет философ. Он рассматривает здесь бунт как \textbf{непременный атрибут человеческого существования, объединяющий личность с другими людьми}. Итогом бунта выступает новый бунт. Угнетенные, превратившись в угнетателей, своим поведением \textbf{подготавливают новый бунт} тех, кого они превращают в угнетенных.

\hfill

Согласно Камю «в этом мире действует, один закон — закон силы, и вдохновляется он волей к власти», которая может реализовываться с помощью насилия.

\hfill

Осмысливая возможности применения насилия в бунте, Камю \textbf{не был сторонником ненасилия}, так как, по его мнению, «абсолютное ненасилие пассивно оправдывает рабство и его ужасы». Но в то же время он \textbf{не был и сторонником чрезмерного насилия}. Мыслитель полагал, что «эти два понятия \textbf{нуждаются в самоограничении ради собственной плодотворности}».

\hfill

У Камю от простого бунта отличается \textbf{метафизический бунт}, представляющий собой «\textbf{восстание человека против всего мироздания}». Такой бунт метафизичен, поскольку \textbf{оспаривает конечные цели людей и вселенной}. В обычном бунте раб протестует против угнетения, «метафизический бунтарь бунтует против удела, уготованного ему как представителю рода человеческого». В метафизическом бунте формула «я бунтую, следовательно, мы существуем», характерная для обычного бунта, меняется на формулу «я бунтую, следовательно, мы одиноки».

\begin{definition}
    Метафизический бунт — восстание человека против всего мироздания, оспаривающее конечные цели людей и вселенной
\end{definition}

\hfill

Логическое \textbf{следствие метафизического бунта — революция}. При этом отличие бунта от революции состоит в том, что «...бунт убивает только людей, тогда как \textbf{революция уничтожает одновременно и людей, и принципы}». По мнению Камю, история человечества знала только бунты, \textbf{революций же пока еще не было}. Он считал, что «если бы один единственный раз свершилась подлинная революция, то истории уже не было бы. Было бы блаженное единство и угомонившаяся смерть».

\hfill

Пределом метафизического бунта является, по Камю, метафизическая революция, в ходе которой во главе мира становятся \textbf{великие инквизиторы}. Великие инквизиторы \textbf{устанавливают на земле царство небесное}. Им по силам то, что оказалось не по силам Богу. Царство небесное на земле как воплощение всеобщего счастья возможно «не благодаря полной свободе выбора между добром и злом, а \textbf{благодаря власти над миром и унификации его}».

\hfill

Развивая эту мысль на основе анализа представлений Ф. Ницше о природе свободы, А. Камю приходит к выводу о том, что «\textbf{абсолютная власть закона не есть свобода, но не большей свободой является абсолютная неподвластность закону}. Расширение возможностей не дает свободы, однако \textbf{отсутствие возможностей есть рабство}. Но и \textbf{анархия тоже рабство}. \textbf{Свобода есть только в том мире, где четко определены как возможное, так и невозможное}». Однако «сегодняшний мир, по всей видимости, может быть только миром господ и рабов». Камю был уверен в том, что «\textbf{господство — это тупик}. Поскольку господин никоим образом не может отказаться от господства и стать рабом, \textbf{вечная участь господ жить неудовлетворенными или быть убитыми}. Роль господина в истории сводится только к тому, чтобы \textbf{возрождать рабское сознание}, единственное, которое творит историю». По мнению философа, «то, что именуют историей, является лишь чередой длительных усилий, предпринимаемых ради обретения подлинной свободы». Иными словами, «... \textbf{история — это история труда и бунта}» людей, стремящихся к свободе и справедливости, которые, согласно Камю, связаны. Он считал, что выбирать одну без другой нельзя. Философ подчеркивает: «Если кто-то лишает вас хлеба, он тем самым лишает вас и свободы. Но если у вас отнимают свободу, то будьте уверены, что и хлеб ваш тоже под угрозой, потому что он зависит уже не от вас и вашей борьбы, а от прихоти хозяина».

\hfill

Он считает буржуазную свободу выдумкой. По мнению Альбера Камю, «\textbf{свобода — дело угнетенных}, и ее традиционными защитниками всегда были выходцы из притесняемого народа».

\hfill

Анализируя перспективы человеческого существования в истории, Камю приходит к неутешительному выводу. По его мнению, \textbf{в истории человеку ничего не остается, как «жить в ней... приноравливаясь ко злобе дня, то есть либо лгать, либо молчать»}.

\hfill

В своих этических воззрениях Камю исходил из того, что \textbf{реализация свободы должна опираться на реалистическую мораль}, так как \textbf{моральный нигилизм губителен}.

\hfill

Формулируя свою нравственную позицию, Альбер Камю писал: «\textbf{мы должны служить справедливости, потому что существование наше устроено несправедливо, должны умножать взращивать счастье и радость, потому что мир наш несчастен}».

\hfill

Философ полагал, что для достижения счастья богатство не обязательно. Он был \textbf{против достижения индивидуального счастья путем принесения несчастья другим}. Согласно Камю, «\textbf{самая большая заслуга человека, чтобы жить в одиночестве и безвестности}».

\hfill

Эстетическое в творчестве философа \textbf{служит выражению этического}. Искусство для него является \textbf{средством обнаружения и описания тревожных явлений жизни}. Оно, с его точки зрения, \textbf{может послужить оздоровлению общества}, так как способно вмешиваться в течение жизни.

\end{flushleft}

\pagebreak
\subsection{Экзистенциальная философия К. Ясперса}

\begin{flushleft}

К. Ясперс считал философию неотъемлемым достоянием людей. Ее цель заключалась в том, \textbf{чтобы возвышать человека}, \textbf{помогать ему осознать свою независимость}. Для того чтобы философия отвечала этой цели, ее надо совершенствовать. По мнению К. Ясперса, философия не тождественна науке, хотя \textbf{наука является помощницей философии}. Исследование ее предмета — «личностно-мировоззренческой проблематики» позволяет философии бесконечно углублять и совершенствовать свои выводы.

Исходным понятием философии К. Ясперса является экзистенция, которая понимается как \textbf{источник мышления и действия в самом человеке}.

\begin{definition}
    Экзистенция — источник мышления и действия в самом человеке
\end{definition}

Экзистенция \textbf{способна проявляться в коммуникации}. Последняя может быть \textbf{неподлинной} и \textbf{подлинной}.

\begin{enumerate}
    \item Коммуникация \textbf{наличного бытия}, или \textbf{неподлинная} коммуникация, характеризует общение людей, осуществляющееся с практической целью
    \item  В \textbf{подлинной} или \textbf{экзистенциальной} коммуникации люди противопоставляют себя миру и другим людям. Условием подлинной коммуникации является \textbf{преодоление одиночества, обезличенности человека, его разобщенности с другими людьми}
\end{enumerate}

При этом возможно подлинное бытие, выступающее в качестве бытия с другими. Его \textbf{достижение происходит на путях преодоления «пограничных ситуаций»}, когда люди испытывают повышенное давление мира. Преодолевая эти ситуации, люди приходят к вере в Бога.

\begin{definition}
    Пограничная ситуация — ситуация, которая неизменна по своей сути, и перед которой человек бессилен
\end{definition}

В своих социально-политических воззрениях К. Ясперс исходил из того, что философия \textbf{не может существовать безотносительно к политике}. Философия должна показать человеку, что \textbf{возможно полное крушение того, чем он жил}. Осознание возможной утраты притягательного \textbf{заставляет человека любить этот мир и окружающих людей}.

\hfill

К. Ясперс \textbf{считал невозможным постижение общественного целого и перспектив его развития}, но он не сомневался в том, что \textbf{общество находится в состоянии кризиса}. Кризис этот носит планетарный характер (это кризис всего человечества), он выражается в \textbf{нивелировании интеллекта людей, в утрате основательности в людях, в росте цинизма, в утрате гуманности, в усилении осознания опасности}. При этом «люди ощущают близость катастрофы, стремятся помочь пониманием, воспитанием, введением реформ. Планируя, они пытаются овладеть ходом событий, восстановить необходимые условия или создать новые. Этот кризис, по Ясперсу, \textbf{связан с вступлением общества в век техники}. По его мнению, в нашем веке люди существуют \textbf{не как индивиды, а как некая масса}. Трагедия современного человека заключается в том, что он превращается в элемент массы, толпы. Этому «омассовлению» людей, согласно Ясперсу, способствует установление антигуманных режимов.

\hfill

В своей концепции философии истории К. Ясперс исходит из неприятия теории культурных циклов, разрабатывавшейся сначала О. Шпенглером, а позднее А. Тойнби, согласно которой культуры независимы друг от друга. Ясперс считал, что «\textbf{человечество имеет единые истоки и общую цель}. Эти истоки и эта цель нам неизвестны, во всяком случае в виде достоверного знания». Однако «все мы, люди, происходим от Адама, все мы связаны родством, созданы Богом по образу и подобию его».

\hfill

В отличие от цели истории человечества, \textbf{ее смысл заключается в единстве}, существенную основу которого составляет то, «что люди встречаются в едином духе всеобщей способности понимания... С наибольшей очевидностью единство находит свое выражение в вере в единого Бога». Однако, по мнению философа, «единство истории как полное единение человечества никогда не будет завершено». Ибо насильственно прикованный к ближайшим целям \textbf{человек лишен способности видения жизни в целом}, хотя он и пытается достичь этого видения.

\end{flushleft}

\pagebreak
\subsection{Список литературы}

\subsubsection{Рекомендуемая литература}

\begin{enumerate}
    \item Васильев В. В., Кротов А. А., Бугай Д. В. История философии. М.: Академический проспект, 2008
    \item Алексеев П., Панин А. Философия. М.: Издательство «Проспект», 2005
    \item Спиркин А. Г. Философия: учебник. М. Гардарики, 2006
\end{enumerate}

\subsubsection{Дополнительная литература}

\begin{enumerate}
    \item Сартр Ж.—П. Экзистенциализм — это гуманизм
\end{enumerate}

\end{document}